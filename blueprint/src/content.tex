% In this file you should put the actual content of the blueprint.
% It will be used both by the web and the print version.
% It should *not* include the \begin{document}
%
% If you want to split the blueprint content into several files then
% the current file can be a simple sequence of \input. Otherwise It
% can start with a \section or \chapter for instance.

\tableofcontents

\chapter{Introduction}

The plan is to formalize definitions from the L-functions and modular forms database (LMFDB) in mathlib, as well as creating some tactics to import relevant data from the LMFDB into mathlib.

The LMFDB contains many objects of interest to mathematicians, many of which are still beyond what can currently be formalized in mathlib. For this reason, we will focus on three main areas: number fields, elliptic curves, and modular forms. In each of these areas, we will formalize the relevant definitions and import data from the LMFDB.

Our first main goal is to formalize relevant definitions used by the LMFDB to uniquely identify objects in the database, i.e. the LMFDB labels.

This is still a rough blueprint, generated from the information contained in the LMFDB. For now, we have roughly organized the definitions by area, with a background chapter containing definitions that are needed but don't quite fit into the three main areas above.

\textbf{Warning}: This blueprint is still a work in progress. In places, the LaTeX is not rendering correctly, but everything has a link back to the LMFDB, so if in doubt, it is worth checking the definitions there. Also, many of the definitions are already formalized, and they should soon have links to the relevant definitions in mathlib.



\section{Background}

In this section we list definitions relevant to general mathematical objects and concepts that appear in the LMFDB.

\input{knowls/af/af}
\input{knowls/af/af.bernoulli_numbers}
\input{knowls/af/af.divisor_function}
\input{knowls/af/af.multiplicative}
\input{knowls/ag/ag.abelian_variety}
\input{knowls/ag/ag.affine_space}
\input{knowls/ag/ag.base_change}
\input{knowls/ag/ag.base_field}
\input{knowls/ag/ag.complex_multiplication}
\input{knowls/ag/ag.curve}
\input{knowls/ag/ag.curve.genus}
\input{knowls/ag/ag.curve.smooth}
\input{knowls/ag/ag.dimension}
\input{knowls/ag/ag.endomorphism_algebra}
\input{knowls/ag/ag.endomorphism_ring}
\input{knowls/ag/ag.geom_endomorphism_ring}
\input{knowls/ag/ag.geom_simple}
\input{knowls/ag/ag.hyperelliptic_curve}
\input{knowls/ag/ag.irreducible}
\input{knowls/ag/ag.jacobian}
\input{knowls/ag/ag.minimal_field}
\input{knowls/ag/ag.mordell_weil}
\input{knowls/ag/ag.projective_space}
\input{knowls/ag/ag.quotient_curve}
\input{knowls/ag/ag.riemann_surface}
\input{knowls/ag/ag.simple}
\input{knowls/ag/ag.singular_point}
\input{knowls/ag/ag.variety}
\input{knowls/alg/alg.binary_operation}
\input{knowls/alg/alg.binary_operation.associative}
\input{knowls/alg/alg.binary_operation.commutative}
\input{knowls/alg/alg.binary_operation.identity}
\input{knowls/alg/alg.binary_operation.inverse}
\input{knowls/alg/alg.symplectic_isomorphism}
\input{knowls/artin/artin}
\input{knowls/artin/artin.conductor}
\input{knowls/artin/artin.number_field}
\input{knowls/artin/artin.parity}
\input{knowls/artin/artin.ramified_primes}
\input{knowls/artin/artin.unramified_primes}
\input{knowls/av/av.isogeny}
\input{knowls/av/av.simple}
\input{knowls/av/av.tate_module}
\input{knowls/av/av.twist}
\input{knowls/character/character.dirichlet}
\input{knowls/character/character.dirichlet.conductor}
\input{knowls/character/character.dirichlet.galois_orbit}
\input{knowls/character/character.dirichlet.galois_orbit_index}
\input{knowls/character/character.dirichlet.galois_orbit_label}
\input{knowls/character/character.dirichlet.induce}
\input{knowls/character/character.dirichlet.minimal}
\input{knowls/character/character.dirichlet.modulus}
\input{knowls/character/character.dirichlet.order}
\input{knowls/character/character.dirichlet.primitive}
\input{knowls/character/character.dirichlet.principal}
\input{knowls/character/character.dirichlet.value_field}
\input{knowls/g2c/g2c.aut_grp}
\input{knowls/g2c/g2c.discriminant}
\input{knowls/g2c/g2c.g2curve}
\input{knowls/g2c/g2c.good_reduction}
\input{knowls/g2c/g2c.minimal_equation}
\input{knowls/gg/gg.galois_group}
\input{knowls/gl2/gl2.borel}
\input{knowls/gl2/gl2.cartan}
\input{knowls/gl2/gl2.exceptional}
\input{knowls/gl2/gl2.index}
\input{knowls/gl2/gl2.level}
\input{knowls/gl2/gl2.nonsplit_cartan}
\input{knowls/gl2/gl2.normalizer_cartan}
\input{knowls/gl2/gl2.normalizer_nonsplit_cartan}
\input{knowls/gl2/gl2.normalizer_split_cartan}
\input{knowls/gl2/gl2.open}
\input{knowls/gl2/gl2.profinite}
\input{knowls/gl2/gl2.split_cartan}
\input{knowls/group/group}
\input{knowls/group/group.abelian}
\input{knowls/group/group.automorphism}
\input{knowls/group/group.characteristic_subgroup}
\input{knowls/group/group.coset}
\input{knowls/group/group.frattini_subgroup}
\input{knowls/group/group.fuchsian.cusps}
\input{knowls/group/group.fuchsian.cusps.width}
\input{knowls/group/group.fuchsian.fundamental_domain}
\input{knowls/group/group.galois.absolute}
\input{knowls/group/group.generators}
\input{knowls/group/group.haar_measure}
\input{knowls/group/group.homomorphism}
\input{knowls/group/group.isomorphism}
\input{knowls/group/group.maximal_subgroup}
\input{knowls/group/group.normal_series}
\input{knowls/group/group.order}
\input{knowls/group/group.presentation}
\input{knowls/group/group.rank}
\input{knowls/group/group.sl2z}
\input{knowls/group/group.subgroup}
\input{knowls/group/group.subgroup.index}
\input{knowls/group/group.subgroup.normal}
\input{knowls/group/group.sylow_subgroup}
\input{knowls/group/group.torsion}
\input{knowls/lf/lf.automorphism_group}
\input{knowls/lf/lf.inertia_group}
\input{knowls/lf/lf.local_field}
\input{knowls/lf/lf.maximal_ideal}
\input{knowls/lf/lf.padic_field}
\input{knowls/lf/lf.residue_field}
\input{knowls/lf/lf.ring_of_integers}
\input{knowls/lf/lf.wild_inertia_group}
\input{knowls/lfunction/lfunction}
\input{knowls/lfunction/lfunction.analytic_rank}
\input{knowls/lfunction/lfunction.arithmetic}
\input{knowls/lfunction/lfunction.central_point}
\input{knowls/lfunction/lfunction.critical_line}
\input{knowls/lfunction/lfunction.dirichlet_series}
\input{knowls/lfunction/lfunction.dual}
\input{knowls/lfunction/lfunction.euler_product}
\input{knowls/lfunction/lfunction.functional_equation}
\input{knowls/lfunction/lfunction.gamma_factor}
\input{knowls/lfunction/lfunction.leading_coeff}
\input{knowls/lfunction/lfunction.normalization}
\input{knowls/lfunction/lfunction.rh}
\input{knowls/lfunction/lfunction.self-dual}
\input{knowls/lfunction/lfunction.sign}
\input{knowls/mf/mf.half_integral_weight.dedekind_eta}
\input{knowls/mf/mf.upper_half_plane}
\input{knowls/modcurve/modcurve}
\input{knowls/modcurve/modcurve.cusps}
\input{knowls/modcurve/modcurve.level_structure}
\input{knowls/modcurve/modcurve.xn}
\input{knowls/ring/ring}
\input{knowls/ring/ring.a-field}
\input{knowls/ring/ring.characteristic}
\input{knowls/ring/ring.dedekind_domain}
\input{knowls/ring/ring.field}
\input{knowls/ring/ring.field_of_fractions}
\input{knowls/ring/ring.fractional_ideal}
\input{knowls/ring/ring.ideal}
\input{knowls/ring/ring.integral}
\input{knowls/ring/ring.integral_domain}
\input{knowls/ring/ring.integrally_closed}
\input{knowls/ring/ring.irreducible}
\input{knowls/ring/ring.maximal_ideal}
\input{knowls/ring/ring.noetherian}
\input{knowls/ring/ring.prime_ideal}
\input{knowls/ring/ring.principal_fractional_ideal}
\input{knowls/ring/ring.unit}
\input{knowls/ring/ring.zero_divisor}
\input{knowls/specialfunction/specialfunction.gamma}
\input{knowls/st_group/st_group.definition}
\input{knowls/st_group/st_group.symplectic_form}
\input{knowls/st_group/st_group.usp}


\chapter{Number fields}

In this section we list definitions relevant to number fields and their invariants. This chapter
contains all of the definitions relating to number fields within the LMFDB. Since this list is quite long
we will first give an overview of some key invariants that should be easy to formalise.


\begin{itemize}
    \item \textbf{Label of a number field}: This requires the degree \ref{nf.degree}, (real) signature \ref{nf.signature},
     abs value of discriminant \ref{nf.abs_discriminant} (and an index which we will ignore for now).
    \item root discriminant \ref{nf.root_discriminant}
    \item Galois root discriminant \ref{nf.galois_root_discriminant}
    \item ramified primes \ref{nf.ramified_primes}
    \item discriminant root field \ref{nf.discriminant_root_field}
    \item automorphism group \ref{nf.galois_group}
    \item monogeneric \ref{nf.monogenic}
    \item inessential primes \ref{nf.inessential_prime}
    \item torsion generator \ref{nf.torsion}
    \item fundamental units \ref{nf.fundamental_units}
    \item regulator \ref{nf.regulator}
    \item itermediate fields \ref{nf.intermediate_fields}
    \item sibling fields \ref{nf.sibling}
    \item frobenius cycle type \ref{nf.frobenius_cycle_types}
    \end{itemize}

Next is the full list of invariants contained in the LMFDB.


\section{Definitions relating to number fields}

\input{knowls/nf/nf}
\input{knowls/nf/nf.abelian}
\input{knowls/nf/nf.abs_discriminant}
\input{knowls/nf/nf.absolute_value}
\input{knowls/nf/nf.arithmetically_equivalent}
\input{knowls/nf/nf.class_number}
\input{knowls/nf/nf.class_number_formula}
\input{knowls/nf/nf.cm_field}
\input{knowls/nf/nf.complex_embedding}
\input{knowls/nf/nf.conductor}
\input{knowls/nf/nf.defining_polynomial}
\input{knowls/nf/nf.degree}
\input{knowls/nf/nf.dirichlet_group}
\input{knowls/nf/nf.discriminant}
\input{knowls/nf/nf.discriminant_root_field}
\input{knowls/nf/nf.embedding}
\input{knowls/nf/nf.frobenius_cycle_types}
\input{knowls/nf/nf.fundamental_units}
\input{knowls/nf/nf.galois_closure}
\input{knowls/nf/nf.galois_group}
\input{knowls/nf/nf.galois_root_discriminant}
\input{knowls/nf/nf.generator}
\input{knowls/nf/nf.ideal_class_group}
\input{knowls/nf/nf.ideal_labels}
\input{knowls/nf/nf.inessential_prime}
\input{knowls/nf/nf.integral}
\input{knowls/nf/nf.integral_basis}
\input{knowls/nf/nf.intermediate_fields}
\input{knowls/nf/nf.is_galois}
\input{knowls/nf/nf.local_algebra}
\input{knowls/nf/nf.maximal_cm_subfield}
\input{knowls/nf/nf.minimal_polynomial}
\input{knowls/nf/nf.minimal_sibling}
\input{knowls/nf/nf.monogenic}
\input{knowls/nf/nf.monomial_order}
\input{knowls/nf/nf.narrow_class_group}
\input{knowls/nf/nf.narrow_class_number}
\input{knowls/nf/nf.nickname}
\input{knowls/nf/nf.order}
\input{knowls/nf/nf.padic_completion}
\input{knowls/nf/nf.place}
\input{knowls/nf/nf.polredabs}
\input{knowls/nf/nf.poly_discriminant}
\input{knowls/nf/nf.prime}
\input{knowls/nf/nf.ramified_primes}
\input{knowls/nf/nf.rank}
\input{knowls/nf/nf.real_embedding}
\input{knowls/nf/nf.reflex_field}
\input{knowls/nf/nf.reflex_reflex_field}
\input{knowls/nf/nf.regulator}
\input{knowls/nf/nf.relative_class_number}
\input{knowls/nf/nf.ring_of_integers}
\input{knowls/nf/nf.root_discriminant}
\input{knowls/nf/nf.separable}
\input{knowls/nf/nf.separable_algebra}
\input{knowls/nf/nf.serre_odlyzko_bound}
\input{knowls/nf/nf.sibling}
\input{knowls/nf/nf.signature}
\input{knowls/nf/nf.stem_field}
\input{knowls/nf/nf.torsion}
\input{knowls/nf/nf.totally_imaginary}
\input{knowls/nf/nf.totally_positive}
\input{knowls/nf/nf.totally_real}
\input{knowls/nf/nf.unit_group}
\input{knowls/nf/nf.unramified_prime}
\input{knowls/nf/nf.weil_height}
\input{knowls/nf/nf.weil_polynomial}
\input{knowls/nf/nf.zk_index}




\chapter{Elliptic curves}
Here we list definitions relevant to elliptic curves over number fields and their invariants.

Here is an overview of some of the invariants we might want to include:

\begin{itemize}
    \item LMFDB label (and maybe also Cremona label): Conductor, isogeny class label and isomorphism class index
    \item abc quality
    \item rank
    \item torsion order
\end{itemize}

Next is the full list of invariants contained in the LMFDB.

\input{knowls/ec/ec}
\input{knowls/ec/ec.additive_reduction}
\input{knowls/ec/ec.analytic_sha_order}
\input{knowls/ec/ec.bad_reduction}
\input{knowls/ec/ec.base_change}
\input{knowls/ec/ec.bsdconjecture}
\input{knowls/ec/ec.canonical_height}
\input{knowls/ec/ec.complex_multiplication}
\input{knowls/ec/ec.conductor}
\input{knowls/ec/ec.discriminant}
\input{knowls/ec/ec.endomorphism}
\input{knowls/ec/ec.endomorphism_ring}
\input{knowls/ec/ec.galois_rep}
\input{knowls/ec/ec.galois_rep_adelic_image}
\input{knowls/ec/ec.galois_rep_modell_image}
\input{knowls/ec/ec.geom_endomorphism_ring}
\input{knowls/ec/ec.global_minimal_model}
\input{knowls/ec/ec.good_ordinary_reduction}
\input{knowls/ec/ec.good_reduction}
\input{knowls/ec/ec.good_supersingular_reduction}
\input{knowls/ec/ec.integral_model}
\input{knowls/ec/ec.invariants}
\input{knowls/ec/ec.isogeny}
\input{knowls/ec/ec.isogeny_class}
\input{knowls/ec/ec.isogeny_class_degree}
\input{knowls/ec/ec.isogeny_graph}
\input{knowls/ec/ec.isogeny_matrix}
\input{knowls/ec/ec.isomorphism}
\input{knowls/ec/ec.j_invariant}
\input{knowls/ec/ec.kodaira_symbol}
\input{knowls/ec/ec.local_data}
\input{knowls/ec/ec.local_minimal_discriminant}
\input{knowls/ec/ec.local_minimal_model}
\input{knowls/ec/ec.maximal_elladic_galois_rep}
\input{knowls/ec/ec.maximal_galois_rep}
\input{knowls/ec/ec.minimal_discriminant}
\input{knowls/ec/ec.mordell_weil_group}
\input{knowls/ec/ec.mordell_weil_theorem}
\input{knowls/ec/ec.multiplicative_reduction}
\input{knowls/ec/ec.mw_generators}
\input{knowls/ec/ec.nonsplit_multiplicative_reduction}
\input{knowls/ec/ec.obstruction_class}
\input{knowls/ec/ec.padic_tate_module}
\input{knowls/ec/ec.period}
\input{knowls/ec/ec.potential_good_reduction}
\input{knowls/ec/ec.q}
\input{knowls/ec/ec.q.abc_quality}
\input{knowls/ec/ec.q.analytic_rank}
\input{knowls/ec/ec.q.analytic_sha_order}
\input{knowls/ec/ec.q.bsdconjecture}
\input{knowls/ec/ec.q.canonical_height}
\input{knowls/ec/ec.q.conductor}
\input{knowls/ec/ec.q.cremona_label}
\input{knowls/ec/ec.q.discriminant}
\input{knowls/ec/ec.q.endomorphism_ring}
\input{knowls/ec/ec.q.faltings_height}
\input{knowls/ec/ec.q.faltings_ratio}
\input{knowls/ec/ec.q.frey}
\input{knowls/ec/ec.q.integral_points}
\input{knowls/ec/ec.q.j_invariant}
\input{knowls/ec/ec.q.kodaira_symbol}
\input{knowls/ec/ec.q.lmfdb_label}
\input{knowls/ec/ec.q.manin_constant}
\input{knowls/ec/ec.q.minimal_twist}
\input{knowls/ec/ec.q.minimal_weierstrass_equation}
\input{knowls/ec/ec.q.modular_degree}
\input{knowls/ec/ec.q.modular_parametrization}
\input{knowls/ec/ec.q.naive_height}
\input{knowls/ec/ec.q.optimal}
\input{knowls/ec/ec.q.period_lattice}
\input{knowls/ec/ec.q.real_period}
\input{knowls/ec/ec.q.reduction_type}
\input{knowls/ec/ec.q.regulator}
\input{knowls/ec/ec.q.semistable}
\input{knowls/ec/ec.q.serre_invariants}
\input{knowls/ec/ec.q.special_value}
\input{knowls/ec/ec.q.szpiro_ratio}
\input{knowls/ec/ec.q.torsion_growth}
\input{knowls/ec/ec.q.torsion_subgroup}
\input{knowls/ec/ec.q_curve}
\input{knowls/ec/ec.rank}
\input{knowls/ec/ec.reduction}
\input{knowls/ec/ec.reduction_type}
\input{knowls/ec/ec.regulator}
\input{knowls/ec/ec.ring}
\input{knowls/ec/ec.scheme}
\input{knowls/ec/ec.semi_global_minimal_model}
\input{knowls/ec/ec.semistable}
\input{knowls/ec/ec.simple_equation}
\input{knowls/ec/ec.special_value}
\input{knowls/ec/ec.split_multiplicative_reduction}
\input{knowls/ec/ec.tamagawa_number}
\input{knowls/ec/ec.torsion_order}
\input{knowls/ec/ec.torsion_subgroup}
\input{knowls/ec/ec.twists}
\input{knowls/ec/ec.weierstrass_coeffs}
\input{knowls/ec/ec.weierstrass_isomorphism}


\section{Modular forms}

\begin{itemize}
    \item label (level, weight, galois orbit of dirichlet character, label of galois orbit of newform, corey label, relative dimension)
    \item coefficient field
    \item Character
    \item Hecke operators
    \item newform/old forms
    \item Petersson inner product
    \item L-function self dual
    \item analytic conductor
    \item dimension
    \item Fricke sign/ Atkin-Lehner signs
    \item inner twists
\end{itemize}

Next is the full list of invariants contained in the LMFDB.

\input{knowls/cmf/cmf}
\input{knowls/cmf/cmf.analytic_conductor}
\input{knowls/cmf/cmf.analytic_rank}
\input{knowls/cmf/cmf.artin_field}
\input{knowls/cmf/cmf.artin_image}
\input{knowls/cmf/cmf.atkin-lehner}
\input{knowls/cmf/cmf.bad_prime}
\input{knowls/cmf/cmf.character}
\input{knowls/cmf/cmf.cm_form}
\input{knowls/cmf/cmf.coefficient_field}
\input{knowls/cmf/cmf.coefficient_ring}
\input{knowls/cmf/cmf.congruence_subgroup}
\input{knowls/cmf/cmf.cusp_form}
\input{knowls/cmf/cmf.decomposition.new.gamma0chi}
\input{knowls/cmf/cmf.defining_polynomial}
\input{knowls/cmf/cmf.dimension}
\input{knowls/cmf/cmf.distinguishing_primes}
\input{knowls/cmf/cmf.dualform}
\input{knowls/cmf/cmf.eisenstein}
\input{knowls/cmf/cmf.embedding}
\input{knowls/cmf/cmf.embedding_label}
\input{knowls/cmf/cmf.eta_quotient}
\input{knowls/cmf/cmf.fouriercoefficients}
\input{knowls/cmf/cmf.fricke}
\input{knowls/cmf/cmf.galois_conjugate}
\input{knowls/cmf/cmf.galois_orbit}
\input{knowls/cmf/cmf.galois_representation}
\input{knowls/cmf/cmf.hecke_operator}
\input{knowls/cmf/cmf.hecke_orbit}
\input{knowls/cmf/cmf.hecke_ring_generators}
\input{knowls/cmf/cmf.heckecharpolys}
\input{knowls/cmf/cmf.inner_twist}
\input{knowls/cmf/cmf.inner_twist_count}
\input{knowls/cmf/cmf.inner_twist_multiplicity}
\input{knowls/cmf/cmf.label}
\input{knowls/cmf/cmf.level}
\input{knowls/cmf/cmf.maximal}
\input{knowls/cmf/cmf.minimal}
\input{knowls/cmf/cmf.minimal_twist}
\input{knowls/cmf/cmf.minus_space}
\input{knowls/cmf/cmf.newform}
\input{knowls/cmf/cmf.newform_subspace}
\input{knowls/cmf/cmf.newspace}
\input{knowls/cmf/cmf.nontrivial_twist}
\input{knowls/cmf/cmf.oldspace}
\input{knowls/cmf/cmf.petersson_scalar_product}
\input{knowls/cmf/cmf.plus_space}
\input{knowls/cmf/cmf.projective_field}
\input{knowls/cmf/cmf.projective_image}
\input{knowls/cmf/cmf.q-expansion}
\input{knowls/cmf/cmf.relative_dimension}
\input{knowls/cmf/cmf.rm_form}
\input{knowls/cmf/cmf.satake_angles}
\input{knowls/cmf/cmf.satake_parameters}
\input{knowls/cmf/cmf.sato_tate}
\input{knowls/cmf/cmf.self_twist}
\input{knowls/cmf/cmf.selfdual}
\input{knowls/cmf/cmf.shimura_correspondence}
\input{knowls/cmf/cmf.space}
\input{knowls/cmf/cmf.space_trace_form}
\input{knowls/cmf/cmf.stark_unit}
\input{knowls/cmf/cmf.sturm_bound}
\input{knowls/cmf/cmf.sturm_bound_gamma1}
\input{knowls/cmf/cmf.subspaces}
\input{knowls/cmf/cmf.trace_bound}
\input{knowls/cmf/cmf.trace_form}
\input{knowls/cmf/cmf.twist}
\input{knowls/cmf/cmf.twist_minimal}
\input{knowls/cmf/cmf.twist_multiplicity}
\input{knowls/cmf/cmf.weight}

