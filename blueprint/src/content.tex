% In this file you should put the actual content of the blueprint.
% It will be used both by the web and the print version.
% It should *not* include the \begin{document}
%
% If you want to split the blueprint content into several files then
% the current file can be a simple sequence of \input. Otherwise It
% can start with a \section or \chapter for instance.



\section{Number Fields}

\begin{itemize}
    \item label (degree, (real) signature, abs value of discriminant, index?)
    \item root discriminant
    \item Galois root discriminant
    \item ramified primes
    \item discriminant root field
    \item automorphism group
    \item monogeneric
    \item index
    \item inessential primes
    \item torsion generator
    \item fundamental units
    \item regulator
    \item itermediate fields
    \item sibling fields
    \item frobenius cycle type
    \item local algebra for ramified primes (maybe, depends on local fields stuff).
    \end{itemize}


\section{Modular forms}

\begin{itemize}
    \item label (level, weight, galois orbit of dirichlet character, label of galois orbit of newform, corey label, relative dimension)
    \item coefficient field
    \item Character
    \item Hecke operators
    \item newform/old forms
    \item Petersson inner product
    \item L-function self dual
    \item analytic conductor
    \item dimension
    \item Fricke sign/ Atkin-Lehner signs
    \item inner twists
\end{itemize}

\section{Elliptic curves}

\begin{itemize}
    \item LMFDB label (and maybe also Cremona label): Conductor, isogeny class label and isomorphism class index
    \item abc quality
    \item rank
    \item torsion order



\end{itemize}



\section{{Background}}
\input{knowls/af/af}
\input{knowls/af/af.bernoulli_numbers}
\input{knowls/af/af.divisor_function}
\input{knowls/af/af.multiplicative}
\input{knowls/ag/ag.abelian_surface}
\input{knowls/ag/ag.abelian_variety}
\input{knowls/ag/ag.affine_space}
\input{knowls/ag/ag.algebraic_group}
\input{knowls/ag/ag.arithmetic_genus}
\input{knowls/ag/ag.aut_group}
\input{knowls/ag/ag.bad_prime}
\input{knowls/ag/ag.base_change}
\input{knowls/ag/ag.base_field}
\input{knowls/ag/ag.bielliptic_curve}
\input{knowls/ag/ag.canonical_height}
\input{knowls/ag/ag.canonical_model}
\input{knowls/ag/ag.cluster_picture}
\input{knowls/ag/ag.cohen_macaulay_type}
\input{knowls/ag/ag.complex_multiplication}
\input{knowls/ag/ag.conductor}
\input{knowls/ag/ag.curve}
\input{knowls/ag/ag.curve.genus}
\input{knowls/ag/ag.curve.smooth}
\input{knowls/ag/ag.cyclic_trigonal}
\input{knowls/ag/ag.dimension}
\input{knowls/ag/ag.endomorphism_algebra}
\input{knowls/ag/ag.endomorphism_ring}
\input{knowls/ag/ag.fq.point_counts}
\input{knowls/ag/ag.geom_component}
\input{knowls/ag/ag.geom_endomorphism_algebra}
\input{knowls/ag/ag.geom_endomorphism_ring}
\input{knowls/ag/ag.geom_simple}
\input{knowls/ag/ag.gonality}
\input{knowls/ag/ag.gonality_geom}
\input{knowls/ag/ag.good_reduction}
\input{knowls/ag/ag.holomorphic_euler_char}
\input{knowls/ag/ag.hyperelliptic_curve}
\input{knowls/ag/ag.irreducible}
\input{knowls/ag/ag.isolated_point}
\input{knowls/ag/ag.jacobian}
\input{knowls/ag/ag.kodaira_dimension}
\input{knowls/ag/ag.minimal_field}
\input{knowls/ag/ag.modcurve.x0}
\input{knowls/ag/ag.modcurve.x1}
\input{knowls/ag/ag.modcurve.xnsp}
\input{knowls/ag/ag.mordell_weil}
\input{knowls/ag/ag.plane_model}
\input{knowls/ag/ag.potential_good_reduction}
\input{knowls/ag/ag.primitive}
\input{knowls/ag/ag.projective_space}
\input{knowls/ag/ag.quotient_curve}
\input{knowls/ag/ag.real_endomorphism_algebra}
\input{knowls/ag/ag.real_geom_endomorphism_algebra}
\input{knowls/ag/ag.real_multiplication}
\input{knowls/ag/ag.regulator}
\input{knowls/ag/ag.riemann_surface}
\input{knowls/ag/ag.sato_tate_group}
\input{knowls/ag/ag.selmer_group}
\input{knowls/ag/ag.shimura_variety}
\input{knowls/ag/ag.simple}
\input{knowls/ag/ag.singular_point}
\input{knowls/ag/ag.splitting_field}
\input{knowls/ag/ag.tate_shafarevich}
\input{knowls/ag/ag.torsor}
\input{knowls/ag/ag.variety}
\input{knowls/alg/alg.bezout_matrix}
\input{knowls/alg/alg.binary_operation}
\input{knowls/alg/alg.binary_operation.associative}
\input{knowls/alg/alg.binary_operation.commutative}
\input{knowls/alg/alg.binary_operation.identity}
\input{knowls/alg/alg.binary_operation.inverse}
\input{knowls/alg/alg.matrix.skew_symmetric}
\input{knowls/alg/alg.symmetric_bilinear_form}
\input{knowls/alg/alg.symplectic_isomorphism}
\input{knowls/artin/artin}
\input{knowls/artin/artin.conductor}
\input{knowls/artin/artin.determinant}
\input{knowls/artin/artin.dimension}
\input{knowls/artin/artin.dimensionone}
\input{knowls/artin/artin.frobenius_schur_indicator}
\input{knowls/artin/artin.galois_orbit}
\input{knowls/artin/artin.gg_quotient}
\input{knowls/artin/artin.label}
\input{knowls/artin/artin.lfunction}
\input{knowls/artin/artin.number_field}
\input{knowls/artin/artin.parity}
\input{knowls/artin/artin.permutation_container}
\input{knowls/artin/artin.projective_field}
\input{knowls/artin/artin.projective_image}
\input{knowls/artin/artin.projective_image_type}
\input{knowls/artin/artin.projective_stem_field}
\input{knowls/artin/artin.ramified_primes}
\input{knowls/artin/artin.root_number}
\input{knowls/artin/artin.search_input}
\input{knowls/artin/artin.stem_field}
\input{knowls/artin/artin.trace_of_complex_conj}
\input{knowls/artin/artin.unramified_primes}
\input{knowls/av/av.aut_group}
\input{knowls/av/av.decomposition}
\input{knowls/av/av.deligne_module}
\input{knowls/av/av.endomorphism_field}
\input{knowls/av/av.endomorphism_ring_conductor}
\input{knowls/av/av.fq.abvar.data}
\input{knowls/av/av.fq.all_polarizations_product}
\input{knowls/av/av.fq.angle_rank}
\input{knowls/av/av.fq.bass_zfv}
\input{knowls/av/av.fq.class_number_1}
\input{knowls/av/av.fq.conductor}
\input{knowls/av/av.fq.conjugate_stable}
\input{knowls/av/av.fq.curve_point_counts}
\input{knowls/av/av.fq.endomorphism_ring_notation}
\input{knowls/av/av.fq.frobenius_angles}
\input{knowls/av/av.fq.frobenius_angles_correctness}
\input{knowls/av/av.fq.frobenius_order}
\input{knowls/av/av.fq.galois_group}
\input{knowls/av/av.fq.group_structure}
\input{knowls/av/av.fq.honda_tate}
\input{knowls/av/av.fq.index_of_order}
\input{knowls/av/av.fq.initial_coefficients}
\input{knowls/av/av.fq.is_zconductor_sum}
\input{knowls/av/av.fq.is_zfvconductor_sum}
\input{knowls/av/av.fq.isogeny_class_size}
\input{knowls/av/av.fq.jacobian}
\input{knowls/av/av.fq.l-polynomial}
\input{knowls/av/av.fq.lmfdb_label}
\input{knowls/av/av.fq.max_cohen_macaulay_type}
\input{knowls/av/av.fq.maximal_zfv}
\input{knowls/av/av.fq.newton_elevation}
\input{knowls/av/av.fq.number_field}
\input{knowls/av/av.fq.one_rational_point}
\input{knowls/av/av.fq.order}
\input{knowls/av/av.fq.ordinary}
\input{knowls/av/av.fq.p_rank}
\input{knowls/av/av.fq.picard_group}
\input{knowls/av/av.fq.search_input}
\input{knowls/av/av.fq.singular_dimensions}
\input{knowls/av/av.fq.singular_primes}
\input{knowls/av/av.fq.supersingular}
\input{knowls/av/av.fq.weak_equivalence_class}
\input{knowls/av/av.fq.weil_polynomial}
\input{knowls/av/av.fq.zfv_index}
\input{knowls/av/av.fq.zfv_real_index}
\input{knowls/av/av.fq.zfv_rel_disc_norm}
\input{knowls/av/av.galois_rep}
\input{knowls/av/av.geometrically_simple}
\input{knowls/av/av.geometrically_squarefree}
\input{knowls/av/av.hyperelliptic_count}
\input{knowls/av/av.is_product}
\input{knowls/av/av.isogeny}
\input{knowls/av/av.isogeny_class}
\input{knowls/av/av.jacobian_count}
\input{knowls/av/av.polarization}
\input{knowls/av/av.potential_toric_rank}
\input{knowls/av/av.princ_polarizable}
\input{knowls/av/av.semiabelian_variety}
\input{knowls/av/av.simple}
\input{knowls/av/av.squarefree}
\input{knowls/av/av.tate_module}
\input{knowls/av/av.theta_divisor}
\input{knowls/av/av.twist}
\input{knowls/av/av.weak_equivalence_class}
\input{knowls/belyi/belyi.abc}
\input{knowls/belyi/belyi.base_field}
\input{knowls/belyi/belyi.degree}
\input{knowls/belyi/belyi.galmap}
\input{knowls/belyi/belyi.genus}
\input{knowls/belyi/belyi.geometry_type}
\input{knowls/belyi/belyi.group}
\input{knowls/belyi/belyi.label}
\input{knowls/belyi/belyi.num_orbits}
\input{knowls/belyi/belyi.orbit_size}
\input{knowls/belyi/belyi.orders}
\input{knowls/belyi/belyi.pass_size}
\input{knowls/belyi/belyi.passport}
\input{knowls/belyi/belyi.permutation_triple}
\input{knowls/belyi/belyi.primitive}
\input{knowls/belyi/belyi.primitivization}
\input{knowls/belyi/belyi.ramification_type}
\input{knowls/belyi/belyi.search_input}
\input{knowls/character/character.dirichlet}
\input{knowls/character/character.dirichlet.basic_properties}
\input{knowls/character/character.dirichlet.conductor}
\input{knowls/character/character.dirichlet.conrey}
\input{knowls/character/character.dirichlet.conrey.conductor}
\input{knowls/character/character.dirichlet.conrey.index}
\input{knowls/character/character.dirichlet.conrey.orbit_label}
\input{knowls/character/character.dirichlet.conrey.order}
\input{knowls/character/character.dirichlet.conrey.parity}
\input{knowls/character/character.dirichlet.conrey.primitive_generator}
\input{knowls/character/character.dirichlet.data}
\input{knowls/character/character.dirichlet.degree}
\input{knowls/character/character.dirichlet.field_cut_out}
\input{knowls/character/character.dirichlet.galois_orbit}
\input{knowls/character/character.dirichlet.galois_orbit_index}
\input{knowls/character/character.dirichlet.galois_orbit_label}
\input{knowls/character/character.dirichlet.gauss_sum}
\input{knowls/character/character.dirichlet.group}
\input{knowls/character/character.dirichlet.group.generators}
\input{knowls/character/character.dirichlet.group.order}
\input{knowls/character/character.dirichlet.group.structure}
\input{knowls/character/character.dirichlet.induce}
\input{knowls/character/character.dirichlet.jacobi_sum}
\input{knowls/character/character.dirichlet.jacobi_symbol}
\input{knowls/character/character.dirichlet.kloosterman_sum}
\input{knowls/character/character.dirichlet.kronecker_symbol}
\input{knowls/character/character.dirichlet.legendre_symbol}
\input{knowls/character/character.dirichlet.minimal}
\input{knowls/character/character.dirichlet.modulus}
\input{knowls/character/character.dirichlet.orbit_data}
\input{knowls/character/character.dirichlet.order}
\input{knowls/character/character.dirichlet.parity}
\input{knowls/character/character.dirichlet.primitive}
\input{knowls/character/character.dirichlet.principal}
\input{knowls/character/character.dirichlet.real}
\input{knowls/character/character.dirichlet.related_fields}
\input{knowls/character/character.dirichlet.search_input}
\input{knowls/character/character.dirichlet.value_field}
\input{knowls/character/character.dirichlet.values}
\input{knowls/character/character.dirichlet.values_on_gens}
\input{knowls/character/character.hecke}
\input{knowls/character/character.unit_group}
\input{knowls/clusterpicture/clusterpicture.data}
\input{knowls/combin/combin.hasse_diagram}
\input{knowls/commensurable/commensurable.group}
% The contents of auto_background.tex, auto_number_fields, auto_elliptic_curves, and auto_modular_forms are generated by the update_knowls.py script and should not be edited manually.

\tableofcontents

\chapter{Introduction}

The plan is to formalize definitions from the L-functions and modular forms database (LMFDB) in mathlib, as well as creating some tactics to import relevant data from the LMFDB into mathlib.

The LMFDB contains many objects of interest to mathematicians, many of which are still beyond what can currently be formalized in mathlib. For this reason, we will focus on three main areas: number fields, elliptic curves, and modular forms. In each of these areas, we will formalize relevant definitions and import data from the LMFDB.

Our first main goal is to formalize relevant definitions used by the LMFDB to uniquely identify objects in the database, i.e. the LMFDB labels.

This is still a rough blueprint, generated from the information contained in the LMFDB. For now, we have roughly organized the definitions by area, with a background chapter containing definitions that are needed but don't quite fit into the three main areas above.

\textbf{Warning}: This blueprint is still a work in progress. In places, the LaTeX is not rendering correctly, but everything has a link back to the LMFDB, so if in doubt, it is worth checking the definitions there. Also, many of the definitions are already formalized, and they should soon have links to the relevant definitions in mathlib.

\chapter{Background}

In this section we list definitions (in no particular order) relevant to general mathematical objects and concepts that appear in the LMFDB.
These are definitions that don't quite fit into the three main areas of number fields, elliptic curves and modular forms, but are still needed to understand the definitions in those areas.
Some are either already in mathlib or  beyond what we can currently formalise in mathlib.

\newpage

\input{knowls/af/af}
\input{knowls/af/af.bernoulli_numbers}
\input{knowls/af/af.divisor_function}
\input{knowls/af/af.multiplicative}
\input{knowls/ag/ag.abelian_variety}
\input{knowls/ag/ag.affine_space}
\input{knowls/ag/ag.base_change}
\input{knowls/ag/ag.base_field}
\input{knowls/ag/ag.complex_multiplication}
\input{knowls/ag/ag.curve}
\input{knowls/ag/ag.curve.genus}
\input{knowls/ag/ag.curve.smooth}
\input{knowls/ag/ag.dimension}
\input{knowls/ag/ag.endomorphism_algebra}
\input{knowls/ag/ag.endomorphism_ring}
\input{knowls/ag/ag.geom_endomorphism_ring}
\input{knowls/ag/ag.geom_simple}
\input{knowls/ag/ag.hyperelliptic_curve}
\input{knowls/ag/ag.irreducible}
\input{knowls/ag/ag.jacobian}
\input{knowls/ag/ag.minimal_field}
\input{knowls/ag/ag.mordell_weil}
\input{knowls/ag/ag.projective_space}
\input{knowls/ag/ag.quotient_curve}
\input{knowls/ag/ag.riemann_surface}
\input{knowls/ag/ag.simple}
\input{knowls/ag/ag.singular_point}
\input{knowls/ag/ag.variety}
\input{knowls/alg/alg.binary_operation}
\input{knowls/alg/alg.binary_operation.associative}
\input{knowls/alg/alg.binary_operation.commutative}
\input{knowls/alg/alg.binary_operation.identity}
\input{knowls/alg/alg.binary_operation.inverse}
\input{knowls/alg/alg.symplectic_isomorphism}
\input{knowls/artin/artin}
\input{knowls/artin/artin.conductor}
\input{knowls/artin/artin.number_field}
\input{knowls/artin/artin.parity}
\input{knowls/artin/artin.ramified_primes}
\input{knowls/artin/artin.unramified_primes}
\input{knowls/av/av.isogeny}
\input{knowls/av/av.simple}
\input{knowls/av/av.tate_module}
\input{knowls/av/av.twist}
\input{knowls/character/character.dirichlet}
\input{knowls/character/character.dirichlet.conductor}
\input{knowls/character/character.dirichlet.galois_orbit}
\input{knowls/character/character.dirichlet.galois_orbit_index}
\input{knowls/character/character.dirichlet.galois_orbit_label}
\input{knowls/character/character.dirichlet.induce}
\input{knowls/character/character.dirichlet.minimal}
\input{knowls/character/character.dirichlet.modulus}
\input{knowls/character/character.dirichlet.order}
\input{knowls/character/character.dirichlet.primitive}
\input{knowls/character/character.dirichlet.principal}
\input{knowls/character/character.dirichlet.value_field}
\input{knowls/g2c/g2c.aut_grp}
\input{knowls/g2c/g2c.discriminant}
\input{knowls/g2c/g2c.g2curve}
\input{knowls/g2c/g2c.good_reduction}
\input{knowls/g2c/g2c.minimal_equation}
\input{knowls/gg/gg.galois_group}
\input{knowls/gl2/gl2.borel}
\input{knowls/gl2/gl2.cartan}
\input{knowls/gl2/gl2.exceptional}
\input{knowls/gl2/gl2.index}
\input{knowls/gl2/gl2.level}
\input{knowls/gl2/gl2.nonsplit_cartan}
\input{knowls/gl2/gl2.normalizer_cartan}
\input{knowls/gl2/gl2.normalizer_nonsplit_cartan}
\input{knowls/gl2/gl2.normalizer_split_cartan}
\input{knowls/gl2/gl2.open}
\input{knowls/gl2/gl2.profinite}
\input{knowls/gl2/gl2.split_cartan}
\input{knowls/group/group}
\input{knowls/group/group.abelian}
\input{knowls/group/group.automorphism}
\input{knowls/group/group.characteristic_subgroup}
\input{knowls/group/group.coset}
\input{knowls/group/group.frattini_subgroup}
\input{knowls/group/group.fuchsian.cusps}
\input{knowls/group/group.fuchsian.cusps.width}
\input{knowls/group/group.fuchsian.fundamental_domain}
\input{knowls/group/group.galois.absolute}
\input{knowls/group/group.generators}
\input{knowls/group/group.haar_measure}
\input{knowls/group/group.homomorphism}
\input{knowls/group/group.isomorphism}
\input{knowls/group/group.maximal_subgroup}
\input{knowls/group/group.normal_series}
\input{knowls/group/group.order}
\input{knowls/group/group.presentation}
\input{knowls/group/group.rank}
\input{knowls/group/group.sl2z}
\input{knowls/group/group.subgroup}
\input{knowls/group/group.subgroup.index}
\input{knowls/group/group.subgroup.normal}
\input{knowls/group/group.sylow_subgroup}
\input{knowls/group/group.torsion}
\input{knowls/lf/lf.automorphism_group}
\input{knowls/lf/lf.inertia_group}
\input{knowls/lf/lf.local_field}
\input{knowls/lf/lf.maximal_ideal}
\input{knowls/lf/lf.padic_field}
\input{knowls/lf/lf.residue_field}
\input{knowls/lf/lf.ring_of_integers}
\input{knowls/lf/lf.wild_inertia_group}
\input{knowls/lfunction/lfunction}
\input{knowls/lfunction/lfunction.analytic_rank}
\input{knowls/lfunction/lfunction.arithmetic}
\input{knowls/lfunction/lfunction.central_point}
\input{knowls/lfunction/lfunction.critical_line}
\input{knowls/lfunction/lfunction.dirichlet_series}
\input{knowls/lfunction/lfunction.dual}
\input{knowls/lfunction/lfunction.euler_product}
\input{knowls/lfunction/lfunction.functional_equation}
\input{knowls/lfunction/lfunction.gamma_factor}
\input{knowls/lfunction/lfunction.leading_coeff}
\input{knowls/lfunction/lfunction.normalization}
\input{knowls/lfunction/lfunction.rh}
\input{knowls/lfunction/lfunction.self-dual}
\input{knowls/lfunction/lfunction.sign}
\input{knowls/mf/mf.half_integral_weight.dedekind_eta}
\input{knowls/mf/mf.upper_half_plane}
\input{knowls/modcurve/modcurve}
\input{knowls/modcurve/modcurve.cusps}
\input{knowls/modcurve/modcurve.level_structure}
\input{knowls/modcurve/modcurve.xn}
\input{knowls/ring/ring}
\input{knowls/ring/ring.a-field}
\input{knowls/ring/ring.characteristic}
\input{knowls/ring/ring.dedekind_domain}
\input{knowls/ring/ring.field}
\input{knowls/ring/ring.field_of_fractions}
\input{knowls/ring/ring.fractional_ideal}
\input{knowls/ring/ring.ideal}
\input{knowls/ring/ring.integral}
\input{knowls/ring/ring.integral_domain}
\input{knowls/ring/ring.integrally_closed}
\input{knowls/ring/ring.irreducible}
\input{knowls/ring/ring.maximal_ideal}
\input{knowls/ring/ring.noetherian}
\input{knowls/ring/ring.prime_ideal}
\input{knowls/ring/ring.principal_fractional_ideal}
\input{knowls/ring/ring.unit}
\input{knowls/ring/ring.zero_divisor}
\input{knowls/specialfunction/specialfunction.gamma}
\input{knowls/st_group/st_group.definition}
\input{knowls/st_group/st_group.symplectic_form}
\input{knowls/st_group/st_group.usp}


\chapter{Number fields}

In this section we list definitions relevant to number fields and their invariants. This chapter
contains all of the definitions relating to number fields within the LMFDB. Since this list is quite long
we will first give an overview of some key invariants that should be easy to formalise.


\begin{itemize}
    \item \textbf{Label of a number field}: This requires the degree \ref{nf.degree}, (real) signature \ref{nf.signature},
     abs value of discriminant \ref{nf.abs_discriminant} (and an index which we will ignore for now).
    \item root discriminant \ref{nf.root_discriminant}
    \item Galois root discriminant \ref{nf.galois_root_discriminant}
    \item ramified primes \ref{nf.ramified_primes}
    \item discriminant root field \ref{nf.discriminant_root_field}
    \item automorphism group \ref{nf.galois_group}
    \item monogeneric \ref{nf.monogenic}
    \item inessential primes \ref{nf.inessential_prime}
    \item torsion generator \ref{nf.torsion}
    \item fundamental units \ref{nf.fundamental_units}
    \item regulator \ref{nf.regulator}
    \item itermediate fields \ref{nf.intermediate_fields}
    \item sibling fields \ref{nf.sibling}
    \item frobenius cycle type \ref{nf.frobenius_cycle_types}
    \end{itemize}

Next is the full list of invariants contained in the LMFDB.


\section{Definitions relating to number fields}

\input{knowls/nf/nf}
\input{knowls/nf/nf.abelian}
\input{knowls/nf/nf.abs_discriminant}
\input{knowls/nf/nf.absolute_value}
\input{knowls/nf/nf.arithmetically_equivalent}
\input{knowls/nf/nf.class_number}
\input{knowls/nf/nf.class_number_formula}
\input{knowls/nf/nf.cm_field}
\input{knowls/nf/nf.complex_embedding}
\input{knowls/nf/nf.conductor}
\input{knowls/nf/nf.defining_polynomial}
\input{knowls/nf/nf.degree}
\input{knowls/nf/nf.dirichlet_group}
\input{knowls/nf/nf.discriminant}
\input{knowls/nf/nf.discriminant_root_field}
\input{knowls/nf/nf.embedding}
\input{knowls/nf/nf.frobenius_cycle_types}
\input{knowls/nf/nf.fundamental_units}
\input{knowls/nf/nf.galois_closure}
\input{knowls/nf/nf.galois_group}
\input{knowls/nf/nf.galois_root_discriminant}
\input{knowls/nf/nf.generator}
\input{knowls/nf/nf.ideal_class_group}
\input{knowls/nf/nf.ideal_labels}
\input{knowls/nf/nf.inessential_prime}
\input{knowls/nf/nf.integral}
\input{knowls/nf/nf.integral_basis}
\input{knowls/nf/nf.intermediate_fields}
\input{knowls/nf/nf.is_galois}
\input{knowls/nf/nf.local_algebra}
\input{knowls/nf/nf.maximal_cm_subfield}
\input{knowls/nf/nf.minimal_polynomial}
\input{knowls/nf/nf.minimal_sibling}
\input{knowls/nf/nf.monogenic}
\input{knowls/nf/nf.monomial_order}
\input{knowls/nf/nf.narrow_class_group}
\input{knowls/nf/nf.narrow_class_number}
\input{knowls/nf/nf.nickname}
\input{knowls/nf/nf.order}
\input{knowls/nf/nf.padic_completion}
\input{knowls/nf/nf.place}
\input{knowls/nf/nf.polredabs}
\input{knowls/nf/nf.poly_discriminant}
\input{knowls/nf/nf.prime}
\input{knowls/nf/nf.ramified_primes}
\input{knowls/nf/nf.rank}
\input{knowls/nf/nf.real_embedding}
\input{knowls/nf/nf.reflex_field}
\input{knowls/nf/nf.reflex_reflex_field}
\input{knowls/nf/nf.regulator}
\input{knowls/nf/nf.relative_class_number}
\input{knowls/nf/nf.ring_of_integers}
\input{knowls/nf/nf.root_discriminant}
\input{knowls/nf/nf.separable}
\input{knowls/nf/nf.separable_algebra}
\input{knowls/nf/nf.serre_odlyzko_bound}
\input{knowls/nf/nf.sibling}
\input{knowls/nf/nf.signature}
\input{knowls/nf/nf.stem_field}
\input{knowls/nf/nf.torsion}
\input{knowls/nf/nf.totally_imaginary}
\input{knowls/nf/nf.totally_positive}
\input{knowls/nf/nf.totally_real}
\input{knowls/nf/nf.unit_group}
\input{knowls/nf/nf.unramified_prime}
\input{knowls/nf/nf.weil_height}
\input{knowls/nf/nf.weil_polynomial}
\input{knowls/nf/nf.zk_index}


\chapter{Elliptic curves}
Here we list definitions relevant to elliptic curves over number fields and their invariants.
Again we a interested in the invariants used to label elliptic curves in the LMFDB.

Here is an overview of some of the invariants we might want to include. Some of these non-trivial to define.

\begin{itemize}
    \item \textbf{LMFDB label (and maybe also Cremona label)}: Conductor \ref{ec.conductor},
      isogeny class label \ref{ec.isogeny_class} and isomorphism class index \ref{ec.isomorphism}
    \item abc quality \ref{ec.q.abc_quality}
    \item rank \ref{ec.rank}
    \item torsion order \ref{ec.torsion_order}
\end{itemize}

Next is the full list of invariants contained in the LMFDB.

\section{Definitions relating to elliptic curves over general number fields}

\input{auto_elliptic_curves}

\chapter{Modular forms}

Here we list definitions relevant to classical modular forms and their invariants. We first want to
focus on invariants that are used to label modular forms in the LMFDB. Anlongside this we want to
define Hecke operators and subspaces of newforms and oldforms.


\begin{itemize}
    \item \textbf{Labels of modular forms} (\ref{cmf.label}): level \ref{cmf.level}, weight \ref{cmf.weight},
    galois orbit of dirichlet character \ref{character.dirichlet.galois_orbit_label} , label of galois orbit of newform \label{cmf.galois_orbit}, Conrey label \ref{character.dirichlet.conrey}, relative dimension \ref{cmf.relative_dimension}
    \item coefficient field \ref{cmf.coefficient_field}
    \item Character \ref{cmf.character}
    \item Hecke operators \ref{cmf.hecke_operator}
    \item newform/old forms \ref{cmf.newform}, \ref{cmf.oldspace}
    \item Petersson inner product \ref{cmf.petersson_scalar_product}
    \item L-function self dual \ref{cmf.selfdual}
    \item analytic conductor \ref{cmf.analytic_conductor}
    \item dimension \ref{cmf.dimension}
    \item Fricke sign/ Atkin-Lehner signs \ref{cmf.fricke}, \ref{cmf.atkin-lehner}
    \item inner twists \ref{cmf.inner_twist}, \ref{cmf.inner_twist_count}, \ref{cmf.inner_twist_multiplicity}
\end{itemize}

Next is the full list of invariants contained in the LMFDB.

\section{Definitions relating to classical modular forms}

\input{auto_modular_forms}

% The contents of auto_background.tex, auto_number_fields, auto_elliptic_curves, and auto_modular_forms are generated by the update_knowls.py script and should not be edited manually.

\tableofcontents

\chapter{Introduction}

The plan is to formalize definitions from the L-functions and modular forms database (LMFDB) in mathlib, as well as creating some tactics to import relevant data from the LMFDB into mathlib.

The LMFDB contains many objects of interest to mathematicians, many of which are still beyond what can currently be formalized in mathlib. For this reason, we will focus on three main areas: number fields, elliptic curves, and modular forms. In each of these areas, we will formalize relevant definitions and import data from the LMFDB.

Our first main goal is to formalize relevant definitions used by the LMFDB to uniquely identify objects in the database, i.e. the LMFDB labels.

This is still a rough blueprint, generated from the information contained in the LMFDB. For now, we have roughly organized the definitions by area, with a background chapter containing definitions that are needed but don't quite fit into the three main areas above.

\textbf{Warning}: This blueprint is still a work in progress. In places, the LaTeX is not rendering correctly, but everything has a link back to the LMFDB, so if in doubt, it is worth checking the definitions there. Also, many of the definitions are already formalized, and they should soon have links to the relevant definitions in mathlib.

\chapter{Background}

In this section we list definitions (in no particular order) relevant to general mathematical objects and concepts that appear in the LMFDB.
These are definitions that don't quite fit into the three main areas of number fields, elliptic curves and modular forms, but are still needed to understand the definitions in those areas.
Some are either already in mathlib or  beyond what we can currently formalise in mathlib.

\newpage

\input{knowls/af/af}
\input{knowls/af/af.bernoulli_numbers}
\input{knowls/af/af.divisor_function}
\input{knowls/af/af.multiplicative}
\input{knowls/ag/ag.abelian_variety}
\input{knowls/ag/ag.affine_space}
\input{knowls/ag/ag.base_change}
\input{knowls/ag/ag.base_field}
\input{knowls/ag/ag.complex_multiplication}
\input{knowls/ag/ag.curve}
\input{knowls/ag/ag.curve.genus}
\input{knowls/ag/ag.curve.smooth}
\input{knowls/ag/ag.dimension}
\input{knowls/ag/ag.endomorphism_algebra}
\input{knowls/ag/ag.endomorphism_ring}
\input{knowls/ag/ag.geom_endomorphism_ring}
\input{knowls/ag/ag.geom_simple}
\input{knowls/ag/ag.hyperelliptic_curve}
\input{knowls/ag/ag.irreducible}
\input{knowls/ag/ag.jacobian}
\input{knowls/ag/ag.minimal_field}
\input{knowls/ag/ag.mordell_weil}
\input{knowls/ag/ag.projective_space}
\input{knowls/ag/ag.quotient_curve}
\input{knowls/ag/ag.riemann_surface}
\input{knowls/ag/ag.simple}
\input{knowls/ag/ag.singular_point}
\input{knowls/ag/ag.variety}
\input{knowls/alg/alg.binary_operation}
\input{knowls/alg/alg.binary_operation.associative}
\input{knowls/alg/alg.binary_operation.commutative}
\input{knowls/alg/alg.binary_operation.identity}
\input{knowls/alg/alg.binary_operation.inverse}
\input{knowls/alg/alg.symplectic_isomorphism}
\input{knowls/artin/artin}
\input{knowls/artin/artin.conductor}
\input{knowls/artin/artin.number_field}
\input{knowls/artin/artin.parity}
\input{knowls/artin/artin.ramified_primes}
\input{knowls/artin/artin.unramified_primes}
\input{knowls/av/av.isogeny}
\input{knowls/av/av.simple}
\input{knowls/av/av.tate_module}
\input{knowls/av/av.twist}
\input{knowls/character/character.dirichlet}
\input{knowls/character/character.dirichlet.conductor}
\input{knowls/character/character.dirichlet.galois_orbit}
\input{knowls/character/character.dirichlet.galois_orbit_index}
\input{knowls/character/character.dirichlet.galois_orbit_label}
\input{knowls/character/character.dirichlet.induce}
\input{knowls/character/character.dirichlet.minimal}
\input{knowls/character/character.dirichlet.modulus}
\input{knowls/character/character.dirichlet.order}
\input{knowls/character/character.dirichlet.primitive}
\input{knowls/character/character.dirichlet.principal}
\input{knowls/character/character.dirichlet.value_field}
\input{knowls/g2c/g2c.aut_grp}
\input{knowls/g2c/g2c.discriminant}
\input{knowls/g2c/g2c.g2curve}
\input{knowls/g2c/g2c.good_reduction}
\input{knowls/g2c/g2c.minimal_equation}
\input{knowls/gg/gg.galois_group}
\input{knowls/gl2/gl2.borel}
\input{knowls/gl2/gl2.cartan}
\input{knowls/gl2/gl2.exceptional}
\input{knowls/gl2/gl2.index}
\input{knowls/gl2/gl2.level}
\input{knowls/gl2/gl2.nonsplit_cartan}
\input{knowls/gl2/gl2.normalizer_cartan}
\input{knowls/gl2/gl2.normalizer_nonsplit_cartan}
\input{knowls/gl2/gl2.normalizer_split_cartan}
\input{knowls/gl2/gl2.open}
\input{knowls/gl2/gl2.profinite}
\input{knowls/gl2/gl2.split_cartan}
\input{knowls/group/group}
\input{knowls/group/group.abelian}
\input{knowls/group/group.automorphism}
\input{knowls/group/group.characteristic_subgroup}
\input{knowls/group/group.coset}
\input{knowls/group/group.frattini_subgroup}
\input{knowls/group/group.fuchsian.cusps}
\input{knowls/group/group.fuchsian.cusps.width}
\input{knowls/group/group.fuchsian.fundamental_domain}
\input{knowls/group/group.galois.absolute}
\input{knowls/group/group.generators}
\input{knowls/group/group.haar_measure}
\input{knowls/group/group.homomorphism}
\input{knowls/group/group.isomorphism}
\input{knowls/group/group.maximal_subgroup}
\input{knowls/group/group.normal_series}
\input{knowls/group/group.order}
\input{knowls/group/group.presentation}
\input{knowls/group/group.rank}
\input{knowls/group/group.sl2z}
\input{knowls/group/group.subgroup}
\input{knowls/group/group.subgroup.index}
\input{knowls/group/group.subgroup.normal}
\input{knowls/group/group.sylow_subgroup}
\input{knowls/group/group.torsion}
\input{knowls/lf/lf.automorphism_group}
\input{knowls/lf/lf.inertia_group}
\input{knowls/lf/lf.local_field}
\input{knowls/lf/lf.maximal_ideal}
\input{knowls/lf/lf.padic_field}
\input{knowls/lf/lf.residue_field}
\input{knowls/lf/lf.ring_of_integers}
\input{knowls/lf/lf.wild_inertia_group}
\input{knowls/lfunction/lfunction}
\input{knowls/lfunction/lfunction.analytic_rank}
\input{knowls/lfunction/lfunction.arithmetic}
\input{knowls/lfunction/lfunction.central_point}
\input{knowls/lfunction/lfunction.critical_line}
\input{knowls/lfunction/lfunction.dirichlet_series}
\input{knowls/lfunction/lfunction.dual}
\input{knowls/lfunction/lfunction.euler_product}
\input{knowls/lfunction/lfunction.functional_equation}
\input{knowls/lfunction/lfunction.gamma_factor}
\input{knowls/lfunction/lfunction.leading_coeff}
\input{knowls/lfunction/lfunction.normalization}
\input{knowls/lfunction/lfunction.rh}
\input{knowls/lfunction/lfunction.self-dual}
\input{knowls/lfunction/lfunction.sign}
\input{knowls/mf/mf.half_integral_weight.dedekind_eta}
\input{knowls/mf/mf.upper_half_plane}
\input{knowls/modcurve/modcurve}
\input{knowls/modcurve/modcurve.cusps}
\input{knowls/modcurve/modcurve.level_structure}
\input{knowls/modcurve/modcurve.xn}
\input{knowls/ring/ring}
\input{knowls/ring/ring.a-field}
\input{knowls/ring/ring.characteristic}
\input{knowls/ring/ring.dedekind_domain}
\input{knowls/ring/ring.field}
\input{knowls/ring/ring.field_of_fractions}
\input{knowls/ring/ring.fractional_ideal}
\input{knowls/ring/ring.ideal}
\input{knowls/ring/ring.integral}
\input{knowls/ring/ring.integral_domain}
\input{knowls/ring/ring.integrally_closed}
\input{knowls/ring/ring.irreducible}
\input{knowls/ring/ring.maximal_ideal}
\input{knowls/ring/ring.noetherian}
\input{knowls/ring/ring.prime_ideal}
\input{knowls/ring/ring.principal_fractional_ideal}
\input{knowls/ring/ring.unit}
\input{knowls/ring/ring.zero_divisor}
\input{knowls/specialfunction/specialfunction.gamma}
\input{knowls/st_group/st_group.definition}
\input{knowls/st_group/st_group.symplectic_form}
\input{knowls/st_group/st_group.usp}


\chapter{Number fields}

In this section we list definitions relevant to number fields and their invariants. This chapter
contains all of the definitions relating to number fields within the LMFDB. Since this list is quite long
we will first give an overview of some key invariants that should be easy to formalise.


\begin{itemize}
    \item \textbf{Label of a number field}: This requires the degree \ref{nf.degree}, (real) signature \ref{nf.signature},
     abs value of discriminant \ref{nf.abs_discriminant} (and an index which we will ignore for now).
    \item root discriminant \ref{nf.root_discriminant}
    \item Galois root discriminant \ref{nf.galois_root_discriminant}
    \item ramified primes \ref{nf.ramified_primes}
    \item discriminant root field \ref{nf.discriminant_root_field}
    \item automorphism group \ref{nf.galois_group}
    \item monogeneric \ref{nf.monogenic}
    \item inessential primes \ref{nf.inessential_prime}
    \item torsion generator \ref{nf.torsion}
    \item fundamental units \ref{nf.fundamental_units}
    \item regulator \ref{nf.regulator}
    \item itermediate fields \ref{nf.intermediate_fields}
    \item sibling fields \ref{nf.sibling}
    \item frobenius cycle type \ref{nf.frobenius_cycle_types}
    \end{itemize}

Next is the full list of invariants contained in the LMFDB.


\section{Definitions relating to number fields}

\input{knowls/nf/nf}
\input{knowls/nf/nf.abelian}
\input{knowls/nf/nf.abs_discriminant}
\input{knowls/nf/nf.absolute_value}
\input{knowls/nf/nf.arithmetically_equivalent}
\input{knowls/nf/nf.class_number}
\input{knowls/nf/nf.class_number_formula}
\input{knowls/nf/nf.cm_field}
\input{knowls/nf/nf.complex_embedding}
\input{knowls/nf/nf.conductor}
\input{knowls/nf/nf.defining_polynomial}
\input{knowls/nf/nf.degree}
\input{knowls/nf/nf.dirichlet_group}
\input{knowls/nf/nf.discriminant}
\input{knowls/nf/nf.discriminant_root_field}
\input{knowls/nf/nf.embedding}
\input{knowls/nf/nf.frobenius_cycle_types}
\input{knowls/nf/nf.fundamental_units}
\input{knowls/nf/nf.galois_closure}
\input{knowls/nf/nf.galois_group}
\input{knowls/nf/nf.galois_root_discriminant}
\input{knowls/nf/nf.generator}
\input{knowls/nf/nf.ideal_class_group}
\input{knowls/nf/nf.ideal_labels}
\input{knowls/nf/nf.inessential_prime}
\input{knowls/nf/nf.integral}
\input{knowls/nf/nf.integral_basis}
\input{knowls/nf/nf.intermediate_fields}
\input{knowls/nf/nf.is_galois}
\input{knowls/nf/nf.local_algebra}
\input{knowls/nf/nf.maximal_cm_subfield}
\input{knowls/nf/nf.minimal_polynomial}
\input{knowls/nf/nf.minimal_sibling}
\input{knowls/nf/nf.monogenic}
\input{knowls/nf/nf.monomial_order}
\input{knowls/nf/nf.narrow_class_group}
\input{knowls/nf/nf.narrow_class_number}
\input{knowls/nf/nf.nickname}
\input{knowls/nf/nf.order}
\input{knowls/nf/nf.padic_completion}
\input{knowls/nf/nf.place}
\input{knowls/nf/nf.polredabs}
\input{knowls/nf/nf.poly_discriminant}
\input{knowls/nf/nf.prime}
\input{knowls/nf/nf.ramified_primes}
\input{knowls/nf/nf.rank}
\input{knowls/nf/nf.real_embedding}
\input{knowls/nf/nf.reflex_field}
\input{knowls/nf/nf.reflex_reflex_field}
\input{knowls/nf/nf.regulator}
\input{knowls/nf/nf.relative_class_number}
\input{knowls/nf/nf.ring_of_integers}
\input{knowls/nf/nf.root_discriminant}
\input{knowls/nf/nf.separable}
\input{knowls/nf/nf.separable_algebra}
\input{knowls/nf/nf.serre_odlyzko_bound}
\input{knowls/nf/nf.sibling}
\input{knowls/nf/nf.signature}
\input{knowls/nf/nf.stem_field}
\input{knowls/nf/nf.torsion}
\input{knowls/nf/nf.totally_imaginary}
\input{knowls/nf/nf.totally_positive}
\input{knowls/nf/nf.totally_real}
\input{knowls/nf/nf.unit_group}
\input{knowls/nf/nf.unramified_prime}
\input{knowls/nf/nf.weil_height}
\input{knowls/nf/nf.weil_polynomial}
\input{knowls/nf/nf.zk_index}


\chapter{Elliptic curves}
Here we list definitions relevant to elliptic curves over number fields and their invariants.
Again we a interested in the invariants used to label elliptic curves in the LMFDB.

Here is an overview of some of the invariants we might want to include. Some of these non-trivial to define.

\begin{itemize}
    \item \textbf{LMFDB label (and maybe also Cremona label)}: Conductor \ref{ec.conductor},
      isogeny class label \ref{ec.isogeny_class} and isomorphism class index \ref{ec.isomorphism}
    \item abc quality \ref{ec.q.abc_quality}
    \item rank \ref{ec.rank}
    \item torsion order \ref{ec.torsion_order}
\end{itemize}

Next is the full list of invariants contained in the LMFDB.

\section{Definitions relating to elliptic curves over general number fields}

\input{auto_elliptic_curves}

\chapter{Modular forms}

Here we list definitions relevant to classical modular forms and their invariants. We first want to
focus on invariants that are used to label modular forms in the LMFDB. Anlongside this we want to
define Hecke operators and subspaces of newforms and oldforms.


\begin{itemize}
    \item \textbf{Labels of modular forms} (\ref{cmf.label}): level \ref{cmf.level}, weight \ref{cmf.weight},
    galois orbit of dirichlet character \ref{character.dirichlet.galois_orbit_label} , label of galois orbit of newform \label{cmf.galois_orbit}, Conrey label \ref{character.dirichlet.conrey}, relative dimension \ref{cmf.relative_dimension}
    \item coefficient field \ref{cmf.coefficient_field}
    \item Character \ref{cmf.character}
    \item Hecke operators \ref{cmf.hecke_operator}
    \item newform/old forms \ref{cmf.newform}, \ref{cmf.oldspace}
    \item Petersson inner product \ref{cmf.petersson_scalar_product}
    \item L-function self dual \ref{cmf.selfdual}
    \item analytic conductor \ref{cmf.analytic_conductor}
    \item dimension \ref{cmf.dimension}
    \item Fricke sign/ Atkin-Lehner signs \ref{cmf.fricke}, \ref{cmf.atkin-lehner}
    \item inner twists \ref{cmf.inner_twist}, \ref{cmf.inner_twist_count}, \ref{cmf.inner_twist_multiplicity}
\end{itemize}

Next is the full list of invariants contained in the LMFDB.

\section{Definitions relating to classical modular forms}

\input{auto_modular_forms}

\input{knowls/crystals/crystals.crystal}
\input{knowls/crystals/crystals.highest_weight}
\input{knowls/crystals/crystals.littelmann_path}
\input{knowls/curve/curve.highergenus.aut.braid_equivalence}
\input{knowls/curve/curve.highergenus.aut.branchpoints}
\input{knowls/curve/curve.highergenus.aut.characters}
\input{knowls/curve/curve.highergenus.aut.conjugacyclasses}
\input{knowls/curve/curve.highergenus.aut.dimension}
\input{knowls/curve/curve.highergenus.aut.full}
\input{knowls/curve/curve.highergenus.aut.generatingvector}
\input{knowls/curve/curve.highergenus.aut.generators}
\input{knowls/curve/curve.highergenus.aut.group_action}
\input{knowls/curve/curve.highergenus.aut.groupalgebradecomp}
\input{knowls/curve/curve.highergenus.aut.monodromy}
\input{knowls/curve/curve.highergenus.aut.quotientgenus}
\input{knowls/curve/curve.highergenus.aut.refinedpassport}
\input{knowls/curve/curve.highergenus.aut.search_input}
\input{knowls/curve/curve.highergenus.aut.signature}
\input{knowls/curve/curve.highergenus.aut.sort_order}
\input{knowls/curve/curve.highergenus.aut.topological_equivalence}
\input{knowls/curve/curve.highergenus.hurwitz_bound}
\input{knowls/doc/doc.create_knowl}
\input{knowls/doc/doc.knowl}
\input{knowls/doc/doc.knowl.annotation_guidelines}
\input{knowls/doc/doc.knowl.context_free}
\input{knowls/doc/doc.knowl.description}
\input{knowls/doc/doc.knowl.guidelines}
\input{knowls/doc/doc.knowl.identifier}
\input{knowls/doc/doc.knowl.implicit}
\input{knowls/doc/doc.knowl.naming_conventions}
\input{knowls/doc/doc.knowl.naming_suggestions}
\input{knowls/doc/doc.knowl.rename}
\input{knowls/doc/doc.lmfdb.contextualize}
\input{knowls/doc/doc.lmfdb.improve}
\input{knowls/doc/doc.macros}
\input{knowls/doc/doc.news.in_the_news}
\input{knowls/doc/doc.related_to}
\input{knowls/doc/doc.search_columns}
\input{knowls/doc/doc.secret}
\input{knowls/doc/doc.select_search_columns}
\input{knowls/doc/doc.upload_data}
\input{knowls/doc/doc.view_knowl}
\input{knowls/doc/doc.wiki}
\input{knowls/dq/dq.av.fq.further_extent}
\input{knowls/dq/dq.belyi.source}
\input{knowls/dq/dq.charmodl.extent}
\input{knowls/dq/dq.cmf.cc_prec}
\input{knowls/dq/dq.curve.highergenus.aut.extent}
\input{knowls/dq/dq.curve.highergenus.aut.label}
\input{knowls/dq/dq.curve.highergenus.aut.reliability}
\input{knowls/dq/dq.curve.highergenus.aut.source}
\input{knowls/dq/dq.ec.source}
\input{knowls/dq/dq.ecnf.extent}
\input{knowls/dq/dq.ecnf.reliability}
\input{knowls/dq/dq.ecnf.source}
\input{knowls/dq/dq.hecke_algebras.extent}
\input{knowls/dq/dq.hgm.source}
\input{knowls/dq/dq.lattice.source}
\input{knowls/dq/dq.mf.bianchi.reliability}
\input{knowls/dq/dq.mf.bianchi.source}
\input{knowls/dq/dq.mf.hilbert.reliability}
\input{knowls/dq/dq.mf.hilbert.source}
\input{knowls/dq/dq.modlmf.extent}
\input{knowls/dq/dq.rep_galois_modl.extent}
\input{knowls/dq/dq.st.source}
\input{knowls/dq/dq.zeros.zeta.howcomputed}
\input{knowls/dq/dq.zeros.zeta.toomany}
\input{knowls/field/field.about}
\input{knowls/field/field.finite.conway_polynomial}
\input{knowls/g2c/g2c.abs_discriminant}
\input{knowls/g2c/g2c.all_rational_points}
\input{knowls/g2c/g2c.analytic_rank}
\input{knowls/g2c/g2c.analytic_sha}
\input{knowls/g2c/g2c.aut_grp}
\input{knowls/g2c/g2c.bad_lfactors}
\input{knowls/g2c/g2c.bsd_invariants}
\input{knowls/g2c/g2c.conditional_mw_group}
\input{knowls/g2c/g2c.conductor}
\input{knowls/g2c/g2c.decomposition}
\input{knowls/g2c/g2c.discriminant}
\input{knowls/g2c/g2c.end_alg}
\input{knowls/g2c/g2c.g2_invariants}
\input{knowls/g2c/g2c.g2curve}
\input{knowls/g2c/g2c.galois_rep}
\input{knowls/g2c/g2c.galois_rep.non_maximal_primes}
\input{knowls/g2c/g2c.galois_rep_image}
\input{knowls/g2c/g2c.galois_rep_modell_image}
\input{knowls/g2c/g2c.geom_aut_grp}
\input{knowls/g2c/g2c.geom_end_alg}
\input{knowls/g2c/g2c.geom_iso_class}
\input{knowls/g2c/g2c.geometric_invariants}
\input{knowls/g2c/g2c.gl2type}
\input{knowls/g2c/g2c.good_lfactors}
\input{knowls/g2c/g2c.good_reduction}
\input{knowls/g2c/g2c.has_square_sha}
\input{knowls/g2c/g2c.hasse_weil_conjecture}
\input{knowls/g2c/g2c.igusa_clebsch_invariants}
\input{knowls/g2c/g2c.igusa_invariants}
\input{knowls/g2c/g2c.invariants}
\input{knowls/g2c/g2c.isogeny_class}
\input{knowls/g2c/g2c.jac_end_lattice}
\input{knowls/g2c/g2c.jac_endomorphisms}
\input{knowls/g2c/g2c.jacobian}
\input{knowls/g2c/g2c.known_rational_points}
\input{knowls/g2c/g2c.label}
\input{knowls/g2c/g2c.lfunction}
\input{knowls/g2c/g2c.local_invariants}
\input{knowls/g2c/g2c.local_root_number}
\input{knowls/g2c/g2c.locally_solvable}
\input{knowls/g2c/g2c.maximal_galois_rep}
\input{knowls/g2c/g2c.minimal_equation}
\input{knowls/g2c/g2c.mordell_weil_rank}
\input{knowls/g2c/g2c.mw_generator}
\input{knowls/g2c/g2c.mw_generator_order}
\input{knowls/g2c/g2c.num_rat_pts}
\input{knowls/g2c/g2c.num_rat_wpts}
\input{knowls/g2c/g2c.paramodular_conjecture}
\input{knowls/g2c/g2c.real_period}
\input{knowls/g2c/g2c.regulator}
\input{knowls/g2c/g2c.search_input}
\input{knowls/g2c/g2c.semistable_reduction}
\input{knowls/g2c/g2c.simple_equation}
\input{knowls/g2c/g2c.st_group}
\input{knowls/g2c/g2c.st_group_identity_component}
\input{knowls/g2c/g2c.tamagawa}
\input{knowls/g2c/g2c.tame_reduction}
\input{knowls/g2c/g2c.torsion}
\input{knowls/g2c/g2c.torsion_order}
\input{knowls/g2c/g2c.two_selmer_rank}
\input{knowls/g2c/g2c.two_torsion_field}
\input{knowls/gg/gg.arithmetically_equiv_input}
\input{knowls/gg/gg.arithmetically_equivalent}
\input{knowls/gg/gg.character_table.data}
\input{knowls/gg/gg.conjugacy_classes}
\input{knowls/gg/gg.conjugacy_classes.data}
\input{knowls/gg/gg.conway_name}
\input{knowls/gg/gg.degree}
\input{knowls/gg/gg.elusive_group}
\input{knowls/gg/gg.extent}
\input{knowls/gg/gg.field_automorphisms}
\input{knowls/gg/gg.galois_group}
\input{knowls/gg/gg.generic_polynomial}
\input{knowls/gg/gg.group_action_invariants}
\input{knowls/gg/gg.index}
\input{knowls/gg/gg.int_modules}
\input{knowls/gg/gg.invariants}
\input{knowls/gg/gg.label}
\input{knowls/gg/gg.malle_a}
\input{knowls/gg/gg.other_representations}
\input{knowls/gg/gg.parity}
\input{knowls/gg/gg.primitive}
\input{knowls/gg/gg.regular_extension}
\input{knowls/gg/gg.resolvents}
\input{knowls/gg/gg.search_input}
\input{knowls/gg/gg.simple_name}
\input{knowls/gg/gg.subfields}
\input{knowls/gg/gg.subnormal_series}
\input{knowls/gg/gg.tnumber}
\input{knowls/gl2/gl2.borel}
\input{knowls/gl2/gl2.cartan}
\input{knowls/gl2/gl2.exceptional}
\input{knowls/gl2/gl2.genus}
\input{knowls/gl2/gl2.index}
\input{knowls/gl2/gl2.label}
\input{knowls/gl2/gl2.level}
\input{knowls/gl2/gl2.nonsplit_cartan}
\input{knowls/gl2/gl2.normalizer_cartan}
\input{knowls/gl2/gl2.normalizer_nonsplit_cartan}
\input{knowls/gl2/gl2.normalizer_split_cartan}
\input{knowls/gl2/gl2.open}
\input{knowls/gl2/gl2.profinite}
\input{knowls/gl2/gl2.split_cartan}
\input{knowls/gl2/gl2.subgroup_data}
\input{knowls/group/group}
\input{knowls/group/group.a_group}
\input{knowls/group/group.abelian}
\input{knowls/group/group.abelian_invariants}
\input{knowls/group/group.abelianization}
\input{knowls/group/group.abelianization_isolabel}
\input{knowls/group/group.about}
\input{knowls/group/group.affine_general_linear}
\input{knowls/group/group.affine_linear_automorphism}
\input{knowls/group/group.affine_special_linear}
\input{knowls/group/group.affine_symplectic}
\input{knowls/group/group.almost_simple}
\input{knowls/group/group.alternating}
\input{knowls/group/group.ambient}
\input{knowls/group/group.ambient_isolabel}
\input{knowls/group/group.aut_phi_ratio}
\input{knowls/group/group.autjugacy_class}
\input{knowls/group/group.autjugate_subgroup}
\input{knowls/group/group.automorphism}
\input{knowls/group/group.center}
\input{knowls/group/group.center_isolabel}
\input{knowls/group/group.central}
\input{knowls/group/group.central_quotient}
\input{knowls/group/group.central_quotient_isolabel}
\input{knowls/group/group.centralizer}
\input{knowls/group/group.characteristic_subgroup}
\input{knowls/group/group.chevalley}
\input{knowls/group/group.chief_series}
\input{knowls/group/group.commutator_isolabel}
\input{knowls/group/group.commutator_length}
\input{knowls/group/group.commutator_subgroup}
\input{knowls/group/group.complete}
\input{knowls/group/group.complex_character_table}
\input{knowls/group/group.computing_subgroup_labels}
\input{knowls/group/group.conjugacy_class}
\input{knowls/group/group.conjugacy_class.power_classes}
\input{knowls/group/group.core}
\input{knowls/group/group.coset}
\input{knowls/group/group.coxeter}
\input{knowls/group/group.cyclic}
\input{knowls/group/group.derived_series}
\input{knowls/group/group.dicyclic}
\input{knowls/group/group.dihedral}
\input{knowls/group/group.direct_product}
\input{knowls/group/group.division}
\input{knowls/group/group.division_computing_labels}
\input{knowls/group/group.division_maximal}
\input{knowls/group/group.division_small_large}
\input{knowls/group/group.element_order}
\input{knowls/group/group.elementary}
\input{knowls/group/group.exponent}
\input{knowls/group/group.families}
\input{knowls/group/group.find_input}
\input{knowls/group/group.fitting_subgroup}
\input{knowls/group/group.frattini_subgroup}
\input{knowls/group/group.frobenius}
\input{knowls/group/group.fuchsian}
\input{knowls/group/group.fuchsian.action}
\input{knowls/group/group.fuchsian.cusps}
\input{knowls/group/group.fuchsian.cusps.width}
\input{knowls/group/group.fuchsian.fundamental_domain}
\input{knowls/group/group.galois.absolute}
\input{knowls/group/group.gassmann_equivalence}
\input{knowls/group/group.general_linear}
\input{knowls/group/group.generalized_quaternion}
\input{knowls/group/group.generators}
\input{knowls/group/group.group_ring}
\input{knowls/group/group.haar_measure}
\input{knowls/group/group.hash}
\input{knowls/group/group.heisenberg}
\input{knowls/group/group.homomorphism}
\input{knowls/group/group.hyperelementary}
\input{knowls/group/group.inner_automorphism}
\input{knowls/group/group.isoclinism}
\input{knowls/group/group.isomorphism}
\input{knowls/group/group.label}
\input{knowls/group/group.label_complex_group_char}
\input{knowls/group/group.label_conjugacy_class}
\input{knowls/group/group.label_rational_group_char}
\input{knowls/group/group.lie_type}
\input{knowls/group/group.linear}
\input{knowls/group/group.linear_automorphism}
\input{knowls/group/group.lower_central_series}
\input{knowls/group/group.matrix_group}
\input{knowls/group/group.maximal_quotient}
\input{knowls/group/group.maximal_subgroup}
\input{knowls/group/group.metabelian}
\input{knowls/group/group.metacyclic}
\input{knowls/group/group.min_complex_irrep_deg}
\input{knowls/group/group.min_complex_rep_deg}
\input{knowls/group/group.min_faithful_linear}
\input{knowls/group/group.min_rational_irrep_deg}
\input{knowls/group/group.min_rational_rep_deg}
\input{knowls/group/group.minimal_normal}
\input{knowls/group/group.mobius_function}
\input{knowls/group/group.monomial}
\input{knowls/group/group.name}
\input{knowls/group/group.nilpotent}
\input{knowls/group/group.nonsplit_product}
\input{knowls/group/group.normal_series}
\input{knowls/group/group.omega}
\input{knowls/group/group.order}
\input{knowls/group/group.order_conjugacy_class}
\input{knowls/group/group.order_factorization}
\input{knowls/group/group.order_stats}
\input{knowls/group/group.orthogonal}
\input{knowls/group/group.other_dihedral}
\input{knowls/group/group.outer_aut}
\input{knowls/group/group.over_subgroup}
\input{knowls/group/group.paramodular}
\input{knowls/group/group.perfect}
\input{knowls/group/group.permutation_degree}
\input{knowls/group/group.permutation_gens}
\input{knowls/group/group.permutation_module}
\input{knowls/group/group.permutation_representation}
\input{knowls/group/group.pgroup}
\input{knowls/group/group.picture_description}
\input{knowls/group/group.presentation}
\input{knowls/group/group.primary_decomposition}
\input{knowls/group/group.proper_subgroup}
\input{knowls/group/group.properties_interdependencies}
\input{knowls/group/group.pseudo_random_elements}
\input{knowls/group/group.psl2r}
\input{knowls/group/group.quasisimple}
\input{knowls/group/group.quotient}
\input{knowls/group/group.quotient_isolabel}
\input{knowls/group/group.quotient_size}
\input{knowls/group/group.radical}
\input{knowls/group/group.rank}
\input{knowls/group/group.rational_character_table}
\input{knowls/group/group.rational_group}
\input{knowls/group/group.repr_explain}
\input{knowls/group/group.representation.center}
\input{knowls/group/group.representation.character}
\input{knowls/group/group.representation.complex_char_deg}
\input{knowls/group/group.representation.cyclotomic_n}
\input{knowls/group/group.representation.faithful}
\input{knowls/group/group.representation.image}
\input{knowls/group/group.representation.irrep}
\input{knowls/group/group.representation.kernel}
\input{knowls/group/group.representation.min_perm_rep}
\input{knowls/group/group.representation.rational_character}
\input{knowls/group/group.representation.schur_index}
\input{knowls/group/group.representation.type}
\input{knowls/group/group.schur_multiplier}
\input{knowls/group/group.semi_dihedral}
\input{knowls/group/group.semidirect_product}
\input{knowls/group/group.simple}
\input{knowls/group/group.size_conjugacy_class}
\input{knowls/group/group.sl2z}
\input{knowls/group/group.sl2z.fundamental_domain}
\input{knowls/group/group.sl2z.genus_subgroup}
\input{knowls/group/group.sl2z.subgroup.gamma0n}
\input{knowls/group/group.sl2z.subgroup.gamma1n}
\input{knowls/group/group.sl2z.subgroup.gamman}
\input{knowls/group/group.small.data}
\input{knowls/group/group.small_group_label}
\input{knowls/group/group.socle}
\input{knowls/group/group.solvable}
\input{knowls/group/group.sort_order}
\input{knowls/group/group.sp2gr}
\input{knowls/group/group.sp2gr.gamma0}
\input{knowls/group/group.sp2gr.gamma1}
\input{knowls/group/group.sp2gr.ppal.congruence}
\input{knowls/group/group.sp2gr.subgroup.congn}
\input{knowls/group/group.special_linear}
\input{knowls/group/group.spinor_norm}
\input{knowls/group/group.sporadic}
\input{knowls/group/group.stem_extension}
\input{knowls/group/group.stem_group}
\input{knowls/group/group.subgroup}
\input{knowls/group/group.subgroup.centralizer}
\input{knowls/group/group.subgroup.complement}
\input{knowls/group/group.subgroup.diagram}
\input{knowls/group/group.subgroup.diagram.lmfdb}
\input{knowls/group/group.subgroup.hall}
\input{knowls/group/group.subgroup.index}
\input{knowls/group/group.subgroup.normal}
\input{knowls/group/group.subgroup.normal_closure}
\input{knowls/group/group.subgroup.normalizer}
\input{knowls/group/group.subgroup.projective_image}
\input{knowls/group/group.subgroup_isolabel}
\input{knowls/group/group.subgroup_label}
\input{knowls/group/group.subgroup_properties_interdependencies}
\input{knowls/group/group.subgroups_beyond_bound}
\input{knowls/group/group.supersolvable}
\input{knowls/group/group.sylow_subgroup}
\input{knowls/group/group.symmetric}
\input{knowls/group/group.symplectic}
\input{knowls/group/group.torsion}
\input{knowls/group/group.transitive_degree}
\input{knowls/group/group.triangle_group}
\input{knowls/group/group.trivial_subgroup}
\input{knowls/group/group.type}
\input{knowls/group/group.under_subgroup}
\input{knowls/group/group.unitary}
\input{knowls/group/group.upper_central_series}
\input{knowls/group/group.weyl_group}
\input{knowls/group/group.wreath_product}
\input{knowls/group/group.z_group}
\input{knowls/gsp4/gsp4.subgroup_data}
\input{knowls/hecke_algebra/hecke_algebra.definition}
\input{knowls/hgm/hgm.bezout_determinant}
\input{knowls/hgm/hgm.bezout_matrix}
\input{knowls/hgm/hgm.bezout_module}
\input{knowls/hgm/hgm.conductor}
\input{knowls/hgm/hgm.cusp_index}
\input{knowls/hgm/hgm.defining_parameter_ppart}
\input{knowls/hgm/hgm.defining_parameter_primetoppart}
\input{knowls/hgm/hgm.defining_parameters}
\input{knowls/hgm/hgm.degree}
\input{knowls/hgm/hgm.familes}
\input{knowls/hgm/hgm.field.label}
\input{knowls/hgm/hgm.good_prime}
\input{knowls/hgm/hgm.hodge_vector}
\input{knowls/hgm/hgm.imprimitivity_index}
\input{knowls/hgm/hgm.levelt_matrices}
\input{knowls/hgm/hgm.local_discriminant}
\input{knowls/hgm/hgm.monodromy}
\input{knowls/hgm/hgm.motive}
\input{knowls/hgm/hgm.rotation_number}
\input{knowls/hgm/hgm.search_input}
\input{knowls/hgm/hgm.signature}
\input{knowls/hgm/hgm.tame}
\input{knowls/hgm/hgm.type}
\input{knowls/hgm/hgm.weight}
\input{knowls/hgm/hgm.wild}
\input{knowls/hgm/hgm.zigzagplot}
\input{knowls/hmsurface/hmsurface.arithmetic_genus}
\input{knowls/hmsurface/hmsurface.component_ideal}
\input{knowls/hmsurface/hmsurface.congruence_subgroup}
\input{knowls/hmsurface/hmsurface.cusps}
\input{knowls/hmsurface/hmsurface.elliptic_point}
\input{knowls/hmsurface/hmsurface.euler_number}
\input{knowls/hmsurface/hmsurface.hmsurface}
\input{knowls/hmsurface/hmsurface.hodge_numbers}
\input{knowls/hmsurface/hmsurface.k2}
\input{knowls/hmsurface/hmsurface.kodaira_dimension}
\input{knowls/hmsurface/hmsurface.label}
\input{knowls/hmsurface/hmsurface.level}
\input{knowls/hmsurface/hmsurface.levelnorm}
\input{knowls/hmsurface/hmsurface.polarization_module}
\input{knowls/hmsurface/hmsurface.search_input}
\input{knowls/hmsurface/hmsurface.standard_hmsurfaces}
\input{knowls/hmsurface/hmsurface.x0}
\input{knowls/hmsurface/hmsurface.x01}
\input{knowls/intro/intro}
\input{knowls/intro/intro.api}
\input{knowls/intro/intro.direct_sql}
\input{knowls/intro/intro.download_search}
\input{knowls/intro/intro.features}
\input{knowls/intro/intro.search}
\input{knowls/intro/intro.tutorial}
\input{knowls/lattice/lattice.automorphism_group}
\input{knowls/lattice/lattice.catalogue_of_lattices}
\input{knowls/lattice/lattice.class_number}
\input{knowls/lattice/lattice.data}
\input{knowls/lattice/lattice.definition}
\input{knowls/lattice/lattice.density}
\input{knowls/lattice/lattice.determinant}
\input{knowls/lattice/lattice.dimension}
\input{knowls/lattice/lattice.dual}
\input{knowls/lattice/lattice.genus}
\input{knowls/lattice/lattice.gram}
\input{knowls/lattice/lattice.group_order}
\input{knowls/lattice/lattice.hermite_number}
\input{knowls/lattice/lattice.history}
\input{knowls/lattice/lattice.isometry}
\input{knowls/lattice/lattice.kissing}
\input{knowls/lattice/lattice.label}
\input{knowls/lattice/lattice.level}
\input{knowls/lattice/lattice.minimal_vector}
\input{knowls/lattice/lattice.name}
\input{knowls/lattice/lattice.normalized_minimal_vector}
\input{knowls/lattice/lattice.postive_definite}
\input{knowls/lattice/lattice.primitive}
\input{knowls/lattice/lattice.root_lattice}
\input{knowls/lattice/lattice.search_input}
\input{knowls/lattice/lattice.theta}
\input{knowls/lattice/lattice.unimodular}
\input{knowls/lf/lf}
\input{knowls/lf/lf.algebra.data}
\input{knowls/lf/lf.associated_inertia}
\input{knowls/lf/lf.automorphism_group}
\input{knowls/lf/lf.defining_polynomial}
\input{knowls/lf/lf.degree}
\input{knowls/lf/lf.discriminant_exponent}
\input{knowls/lf/lf.discriminant_root_field}
\input{knowls/lf/lf.eisenstein_diagram}
\input{knowls/lf/lf.eisenstein_form}
\input{knowls/lf/lf.eisenstein_polynomial}
\input{knowls/lf/lf.family_ambiguity}
\input{knowls/lf/lf.family_base}
\input{knowls/lf/lf.family_diagrams}
\input{knowls/lf/lf.family_field_count}
\input{knowls/lf/lf.family_invariants}
\input{knowls/lf/lf.family_label}
\input{knowls/lf/lf.family_mass}
\input{knowls/lf/lf.family_polynomial}
\input{knowls/lf/lf.family_varying}
\input{knowls/lf/lf.field.data}
\input{knowls/lf/lf.field.label}
\input{knowls/lf/lf.field_label}
\input{knowls/lf/lf.galois_closure_invariants}
\input{knowls/lf/lf.galois_invariants}
\input{knowls/lf/lf.galois_mean_slope}
\input{knowls/lf/lf.galois_splitting_model}
\input{knowls/lf/lf.heights}
\input{knowls/lf/lf.herbrand_function}
\input{knowls/lf/lf.herbrand_input}
\input{knowls/lf/lf.herbrand_invariant}
\input{knowls/lf/lf.hidden_slopes}
\input{knowls/lf/lf.indices_of_inseparability}
\input{knowls/lf/lf.inertia_group}
\input{knowls/lf/lf.inertia_group_search}
\input{knowls/lf/lf.intermediate_fields}
\input{knowls/lf/lf.invariants}
\input{knowls/lf/lf.jump_set}
\input{knowls/lf/lf.local_field}
\input{knowls/lf/lf.log}
\input{knowls/lf/lf.maximal_ideal}
\input{knowls/lf/lf.means}
\input{knowls/lf/lf.newton_polygon}
\input{knowls/lf/lf.newton_slopes}
\input{knowls/lf/lf.packet}
\input{knowls/lf/lf.padic_field}
\input{knowls/lf/lf.qp}
\input{knowls/lf/lf.ramification_index}
\input{knowls/lf/lf.ramification_polygon}
\input{knowls/lf/lf.ramification_polygon_display}
\input{knowls/lf/lf.rams}
\input{knowls/lf/lf.residual_polynomials}
\input{knowls/lf/lf.residue_field}
\input{knowls/lf/lf.residue_field_degree}
\input{knowls/lf/lf.ring_of_integers}
\input{knowls/lf/lf.root_number}
\input{knowls/lf/lf.roots_of_unity}
\input{knowls/lf/lf.search_input}
\input{knowls/lf/lf.slope_content}
\input{knowls/lf/lf.slopes}
\input{knowls/lf/lf.subfamily}
\input{knowls/lf/lf.swan_slopes}
\input{knowls/lf/lf.tame_degree}
\input{knowls/lf/lf.top_slope}
\input{knowls/lf/lf.unramified_degree}
\input{knowls/lf/lf.unramified_subfield}
\input{knowls/lf/lf.unramified_totally_ramified_tower}
\input{knowls/lf/lf.visible_slopes}
\input{knowls/lf/lf.wild_inertia_group}
\input{knowls/lf/lf.wild_inertia_group_search}
\input{knowls/lf/lf.wild_segments}
\input{knowls/lf/lf.wild_slopes}
\input{knowls/lfunction/lfunction}
\input{knowls/lfunction/lfunction.analytic_conductor}
\input{knowls/lfunction/lfunction.analytic_rank}
\input{knowls/lfunction/lfunction.arithmetic}
\input{knowls/lfunction/lfunction.bad_prime}
\input{knowls/lfunction/lfunction.central_character}
\input{knowls/lfunction/lfunction.central_point}
\input{knowls/lfunction/lfunction.central_value}
\input{knowls/lfunction/lfunction.coefficient_field}
\input{knowls/lfunction/lfunction.completed}
\input{knowls/lfunction/lfunction.conductor}
\input{knowls/lfunction/lfunction.critical_line}
\input{knowls/lfunction/lfunction.critical_strip}
\input{knowls/lfunction/lfunction.degree}
\input{knowls/lfunction/lfunction.degree2holo.key}
\input{knowls/lfunction/lfunction.dirichlet}
\input{knowls/lfunction/lfunction.dirichlet_series}
\input{knowls/lfunction/lfunction.dual}
\input{knowls/lfunction/lfunction.euler_product}
\input{knowls/lfunction/lfunction.functional_equation}
\input{knowls/lfunction/lfunction.gamma_factor}
\input{knowls/lfunction/lfunction.history}
\input{knowls/lfunction/lfunction.history.artin}
\input{knowls/lfunction/lfunction.history.dedekind}
\input{knowls/lfunction/lfunction.history.dirichlet}
\input{knowls/lfunction/lfunction.history.euler}
\input{knowls/lfunction/lfunction.history.hasse_weil}
\input{knowls/lfunction/lfunction.history.hecke_characters}
\input{knowls/lfunction/lfunction.history.hecke_operators}
\input{knowls/lfunction/lfunction.history.langlands_program}
\input{knowls/lfunction/lfunction.history.maass}
\input{knowls/lfunction/lfunction.history.ramanujan_tau}
\input{knowls/lfunction/lfunction.history.rankin_selberg}
\input{knowls/lfunction/lfunction.history.riemanns_memoir}
\input{knowls/lfunction/lfunction.history.siegel}
\input{knowls/lfunction/lfunction.history.tates_thesis}
\input{knowls/lfunction/lfunction.infinity_factor}
\input{knowls/lfunction/lfunction.invariants}
\input{knowls/lfunction/lfunction.known_degree1}
\input{knowls/lfunction/lfunction.known_degree2}
\input{knowls/lfunction/lfunction.known_degree3}
\input{knowls/lfunction/lfunction.known_degree4}
\input{knowls/lfunction/lfunction.label}
\input{knowls/lfunction/lfunction.leading_coeff}
\input{knowls/lfunction/lfunction.motivic_weight}
\input{knowls/lfunction/lfunction.normalization}
\input{knowls/lfunction/lfunction.primitive}
\input{knowls/lfunction/lfunction.rational}
\input{knowls/lfunction/lfunction.rh}
\input{knowls/lfunction/lfunction.riemann}
\input{knowls/lfunction/lfunction.riemann.euler_product}
\input{knowls/lfunction/lfunction.root_analytic_conductor}
\input{knowls/lfunction/lfunction.root_angle}
\input{knowls/lfunction/lfunction.root_angle_input}
\input{knowls/lfunction/lfunction.root_number}
\input{knowls/lfunction/lfunction.search_input}
\input{knowls/lfunction/lfunction.selberg_class}
\input{knowls/lfunction/lfunction.selberg_class.axioms}
\input{knowls/lfunction/lfunction.selbergdata}
\input{knowls/lfunction/lfunction.self-dual}
\input{knowls/lfunction/lfunction.sign}
\input{knowls/lfunction/lfunction.signature}
\input{knowls/lfunction/lfunction.spectral_label}
\input{knowls/lfunction/lfunction.spectral_parameters}
\input{knowls/lfunction/lfunction.symm}
\input{knowls/lfunction/lfunction.taxonomy}
\input{knowls/lfunction/lfunction.trivial_zero}
\input{knowls/lfunction/lfunction.underlying_object}
\input{knowls/lfunction/lfunction.zeros}
\input{knowls/lfunction/lfunction.zfunction}
\input{knowls/lmfdb/lmfdb.object_information}
\input{knowls/lucant/lucant}
\input{knowls/mf/mf}
\input{knowls/mf/mf.about}
\input{knowls/mf/mf.base_change}
\input{knowls/mf/mf.bianchi}
\input{knowls/mf/mf.bianchi.anr}
\input{knowls/mf/mf.bianchi.base_change}
\input{knowls/mf/mf.bianchi.bianchicongruencesubgroup}
\input{knowls/mf/mf.bianchi.bianchigroup}
\input{knowls/mf/mf.bianchi.cm}
\input{knowls/mf/mf.bianchi.fourierbessell}
\input{knowls/mf/mf.bianchi.hecke_algebra}
\input{knowls/mf/mf.bianchi.hyperbolic3space}
\input{knowls/mf/mf.bianchi.labels}
\input{knowls/mf/mf.bianchi.level}
\input{knowls/mf/mf.bianchi.newform}
\input{knowls/mf/mf.bianchi.search_input}
\input{knowls/mf/mf.bianchi.sign}
\input{knowls/mf/mf.bianchi.spaces}
\input{knowls/mf/mf.bianchi.weight}
\input{knowls/mf/mf.bianchi.weight2}
\input{knowls/mf/mf.cm}
\input{knowls/mf/mf.ellitpic.self_twist}
\input{knowls/mf/mf.gl2.history.combinatorics}
\input{knowls/mf/mf.gl2.history.elliptic}
\input{knowls/mf/mf.gl2.history.hecke}
\input{knowls/mf/mf.gl2.history.infinite}
\input{knowls/mf/mf.gl2.history.new}
\input{knowls/mf/mf.gl2.history.poincare}
\input{knowls/mf/mf.gl2.history.remainder}
\input{knowls/mf/mf.gl2.history.theta}
\input{knowls/mf/mf.gl2.history.varieties}
\input{knowls/mf/mf.growth_condition}
\input{knowls/mf/mf.half_integral_weight}
\input{knowls/mf/mf.half_integral_weight.dedekind_eta}
\input{knowls/mf/mf.half_integral_weight.shimura_decomposition}
\input{knowls/mf/mf.half_integral_weight.theta}
\input{knowls/mf/mf.hilbert}
\input{knowls/mf/mf.hilbert.dimension}
\input{knowls/mf/mf.hilbert.eigenvalue}
\input{knowls/mf/mf.hilbert.hecke_orbit}
\input{knowls/mf/mf.hilbert.label}
\input{knowls/mf/mf.hilbert.level_norm}
\input{knowls/mf/mf.hilbert.q_expansion}
\input{knowls/mf/mf.hilbert.search_input}
\input{knowls/mf/mf.hilbert.weight_vector}
\input{knowls/mf/mf.jacobi}
\input{knowls/mf/mf.maass}
\input{knowls/mf/mf.maass.exceptional_eigenvalue}
\input{knowls/mf/mf.maass.label}
\input{knowls/mf/mf.maass.mwf}
\input{knowls/mf/mf.maass.mwf.character}
\input{knowls/mf/mf.maass.mwf.coefficients}
\input{knowls/mf/mf.maass.mwf.dimension}
\input{knowls/mf/mf.maass.mwf.eigenvalue}
\input{knowls/mf/mf.maass.mwf.fourierexpansion}
\input{knowls/mf/mf.maass.mwf.laplacian}
\input{knowls/mf/mf.maass.mwf.level}
\input{knowls/mf/mf.maass.mwf.ncoefficients}
\input{knowls/mf/mf.maass.mwf.plot}
\input{knowls/mf/mf.maass.mwf.precision}
\input{knowls/mf/mf.maass.mwf.spectralparameter}
\input{knowls/mf/mf.maass.mwf.symmetry}
\input{knowls/mf/mf.maass.mwf.weight}
\input{knowls/mf/mf.maass.mwf.weight0}
\input{knowls/mf/mf.maass.picard}
\input{knowls/mf/mf.maass.picard.eigenvalue}
\input{knowls/mf/mf.maass.picard.fourierexpansion}
\input{knowls/mf/mf.maass.picard.laplacian}
\input{knowls/mf/mf.maass.spectral_index}
\input{knowls/mf/mf.multipliersystem}
\input{knowls/mf/mf.quasi_modular}
\input{knowls/mf/mf.siegel}
\input{knowls/mf/mf.siegel.automorphic_type}
\input{knowls/mf/mf.siegel.character}
\input{knowls/mf/mf.siegel.coefficient_field}
\input{knowls/mf/mf.siegel.coefficient_ring}
\input{knowls/mf/mf.siegel.cusp_form}
\input{knowls/mf/mf.siegel.cusp_form_degree2}
\input{knowls/mf/mf.siegel.defining_polynomial}
\input{knowls/mf/mf.siegel.degree}
\input{knowls/mf/mf.siegel.dimension}
\input{knowls/mf/mf.siegel.eisenstein_series}
\input{knowls/mf/mf.siegel.extent.kp}
\input{knowls/mf/mf.siegel.extent.sp4z}
\input{knowls/mf/mf.siegel.extent.sp4z_2}
\input{knowls/mf/mf.siegel.extent.sp6z}
\input{knowls/mf/mf.siegel.extent.sp8z}
\input{knowls/mf/mf.siegel.family}
\input{knowls/mf/mf.siegel.family.gamma0_2}
\input{knowls/mf/mf.siegel.family.gamma0_3}
\input{knowls/mf/mf.siegel.family.gamma0_3_psi_3}
\input{knowls/mf/mf.siegel.family.gamma0_4}
\input{knowls/mf/mf.siegel.family.gamma0_4_psi_4}
\input{knowls/mf/mf.siegel.family.gamma_2}
\input{knowls/mf/mf.siegel.family.kp}
\input{knowls/mf/mf.siegel.family.sp4z}
\input{knowls/mf/mf.siegel.family.sp4z_2}
\input{knowls/mf/mf.siegel.family.sp6z}
\input{knowls/mf/mf.siegel.family.sp8z}
\input{knowls/mf/mf.siegel.galois_orbit}
\input{knowls/mf/mf.siegel.halfspace}
\input{knowls/mf/mf.siegel.hecke_operator}
\input{knowls/mf/mf.siegel.klingen_eisenstein_series}
\input{knowls/mf/mf.siegel.koecher.principle}
\input{knowls/mf/mf.siegel.label}
\input{knowls/mf/mf.siegel.level}
\input{knowls/mf/mf.siegel.lift}
\input{knowls/mf/mf.siegel.lift.gritsenko}
\input{knowls/mf/mf.siegel.lift.ikeda}
\input{knowls/mf/mf.siegel.lift.maass}
\input{knowls/mf/mf.siegel.lift.miyawaki}
\input{knowls/mf/mf.siegel.lift.miyawaki1}
\input{knowls/mf/mf.siegel.lift.miyawaki2}
\input{knowls/mf/mf.siegel.lift.saito_kurokawa}
\input{knowls/mf/mf.siegel.newform}
\input{knowls/mf/mf.siegel.newform.paramodular}
\input{knowls/mf/mf.siegel.newform_subspace}
\input{knowls/mf/mf.siegel.newspace}
\input{knowls/mf/mf.siegel.phi}
\input{knowls/mf/mf.siegel.q-expansion}
\input{knowls/mf/mf.siegel.slash.op}
\input{knowls/mf/mf.siegel.subspace.degree.two}
\input{knowls/mf/mf.siegel.theta_constant}
\input{knowls/mf/mf.siegel.theta_series}
\input{knowls/mf/mf.siegel.vector_valued}
\input{knowls/mf/mf.siegel.vector_valued_degree_two}
\input{knowls/mf/mf.siegel.weight}
\input{knowls/mf/mf.siegel.weight_k_j}
\input{knowls/mf/mf.sl2.subgroup.gamma0n}
\input{knowls/mf/mf.sl2.subgroup.gamma1n}
\input{knowls/mf/mf.slash_action}
\input{knowls/mf/mf.transformation_property}
\input{knowls/mf/mf.upper_half_plane}
\input{knowls/modcurve/modcurve}
\input{knowls/modcurve/modcurve.agreeable}
\input{knowls/modcurve/modcurve.canonical_field}
\input{knowls/modcurve/modcurve.cm_discriminants}
\input{knowls/modcurve/modcurve.components}
\input{knowls/modcurve/modcurve.contains_negative_one}
\input{knowls/modcurve/modcurve.cusp_orbits}
\input{knowls/modcurve/modcurve.cusp_widths}
\input{knowls/modcurve/modcurve.cusps}
\input{knowls/modcurve/modcurve.decomposition}
\input{knowls/modcurve/modcurve.elliptic_curve_of_point}
\input{knowls/modcurve/modcurve.elliptic_points}
\input{knowls/modcurve/modcurve.embedded_model}
\input{knowls/modcurve/modcurve.entanglement}
\input{knowls/modcurve/modcurve.entanglement_index}
\input{knowls/modcurve/modcurve.fiber_product}
\input{knowls/modcurve/modcurve.gassmann_class}
\input{knowls/modcurve/modcurve.genus}
\input{knowls/modcurve/modcurve.genus_minus_rank}
\input{knowls/modcurve/modcurve.gonality}
\input{knowls/modcurve/modcurve.index}
\input{knowls/modcurve/modcurve.invariants}
\input{knowls/modcurve/modcurve.isolated_point}
\input{knowls/modcurve/modcurve.j_invariant_map}
\input{knowls/modcurve/modcurve.known_points}
\input{knowls/modcurve/modcurve.label}
\input{knowls/modcurve/modcurve.level}
\input{knowls/modcurve/modcurve.level_structure}
\input{knowls/modcurve/modcurve.local_obstruction}
\input{knowls/modcurve/modcurve.minimal_twist}
\input{knowls/modcurve/modcurve.model}
\input{knowls/modcurve/modcurve.models}
\input{knowls/modcurve/modcurve.modular_cover}
\input{knowls/modcurve/modcurve.name}
\input{knowls/modcurve/modcurve.newform_level}
\input{knowls/modcurve/modcurve.nonrational_point}
\input{knowls/modcurve/modcurve.other_labels}
\input{knowls/modcurve/modcurve.plane_model}
\input{knowls/modcurve/modcurve.point_degree}
\input{knowls/modcurve/modcurve.point_residue_field}
\input{knowls/modcurve/modcurve.psl2index}
\input{knowls/modcurve/modcurve.quadratic_refinements}
\input{knowls/modcurve/modcurve.rank}
\input{knowls/modcurve/modcurve.rational_points}
\input{knowls/modcurve/modcurve.relative_index}
\input{knowls/modcurve/modcurve.search_input}
\input{knowls/modcurve/modcurve.simple}
\input{knowls/modcurve/modcurve.sl2level}
\input{knowls/modcurve/modcurve.standard}
\input{knowls/modcurve/modcurve.x0}
\input{knowls/modcurve/modcurve.x0.remarks}
\input{knowls/modcurve/modcurve.x1}
\input{knowls/modcurve/modcurve.x1.remarks}
\input{knowls/modcurve/modcurve.x1mn}
\input{knowls/modcurve/modcurve.x1mn.remarks}
\input{knowls/modcurve/modcurve.xarith}
\input{knowls/modcurve/modcurve.xarith.remarks}
\input{knowls/modcurve/modcurve.xfull}
\input{knowls/modcurve/modcurve.xfull.remarks}
\input{knowls/modcurve/modcurve.xn}
\input{knowls/modcurve/modcurve.xns}
\input{knowls/modcurve/modcurve.xns.remarks}
\input{knowls/modcurve/modcurve.xns_plus}
\input{knowls/modcurve/modcurve.xns_plus.remarks}
\input{knowls/modcurve/modcurve.xpm1}
\input{knowls/modcurve/modcurve.xpm1.remarks}
\input{knowls/modcurve/modcurve.xpm1mn}
\input{knowls/modcurve/modcurve.xpm1mn.remarks}
\input{knowls/modcurve/modcurve.xs4}
\input{knowls/modcurve/modcurve.xs4.remarks}
\input{knowls/modcurve/modcurve.xsp}
\input{knowls/modcurve/modcurve.xsp.remarks}
\input{knowls/modcurve/modcurve.xsp_plus}
\input{knowls/modcurve/modcurve.xsp_plus.remarks}
\input{knowls/modlgal/modlgal}
\input{knowls/modlgal/modlgal.abelian}
\input{knowls/modlgal/modlgal.absolute_frobenius}
\input{knowls/modlgal/modlgal.absolutely_irreducible}
\input{knowls/modlgal/modlgal.base_ring_characteristic}
\input{knowls/modlgal/modlgal.characteristic}
\input{knowls/modlgal/modlgal.codomain}
\input{knowls/modlgal/modlgal.conductor}
\input{knowls/modlgal/modlgal.det_surjective}
\input{knowls/modlgal/modlgal.determinant}
\input{knowls/modlgal/modlgal.determinant_index}
\input{knowls/modlgal/modlgal.dimension}
\input{knowls/modlgal/modlgal.dual_pair_of_algebras}
\input{knowls/modlgal/modlgal.frobenius_charpoly}
\input{knowls/modlgal/modlgal.frobenius_determinant}
\input{knowls/modlgal/modlgal.frobenius_matrix}
\input{knowls/modlgal/modlgal.frobenius_order}
\input{knowls/modlgal/modlgal.frobenius_prime}
\input{knowls/modlgal/modlgal.frobenius_trace}
\input{knowls/modlgal/modlgal.generating_primes}
\input{knowls/modlgal/modlgal.good_prime}
\input{knowls/modlgal/modlgal.image}
\input{knowls/modlgal/modlgal.image_abstract_group}
\input{knowls/modlgal/modlgal.image_index}
\input{knowls/modlgal/modlgal.image_order}
\input{knowls/modlgal/modlgal.label}
\input{knowls/modlgal/modlgal.min_sib_splitting_field}
\input{knowls/modlgal/modlgal.projective_kernel_polynomial}
\input{knowls/modlgal/modlgal.projective_representation}
\input{knowls/modlgal/modlgal.ramified}
\input{knowls/modlgal/modlgal.search_input}
\input{knowls/modlgal/modlgal.solvable}
\input{knowls/modlgal/modlgal.splitting_field}
\input{knowls/modlgal/modlgal.surjective}
\input{knowls/modlgal/modlgal.top_slope}
\input{knowls/mot/mot.hodgevector}
\input{knowls/mot/mot.width}
\input{knowls/motives/motives.about}
\input{knowls/paramodulargroup/paramodulargroup.sp2gr}
\input{knowls/portrait/portrait.cmf}
\input{knowls/portrait/portrait.ec.q}
\input{knowls/portrait/portrait.gg}
\input{knowls/portrait/portrait.groups.abstract}
\input{knowls/portrait/portrait.maass}
\input{knowls/portrait/portrait.modcurve}
\input{knowls/rcs/rcs}
\input{knowls/rcs/rcs.ack.artin}
\input{knowls/rcs/rcs.ack.av.fq}
\input{knowls/rcs/rcs.ack.belyi}
\input{knowls/rcs/rcs.ack.character.dirichlet}
\input{knowls/rcs/rcs.ack.cmf}
\input{knowls/rcs/rcs.ack.curve.highergenus.aut}
\input{knowls/rcs/rcs.ack.ec}
\input{knowls/rcs/rcs.ack.ec.q}
\input{knowls/rcs/rcs.ack.g2c}
\input{knowls/rcs/rcs.ack.general}
\input{knowls/rcs/rcs.ack.gg}
\input{knowls/rcs/rcs.ack.groups.abstract}
\input{knowls/rcs/rcs.ack.hgm}
\input{knowls/rcs/rcs.ack.hosting}
\input{knowls/rcs/rcs.ack.lattice}
\input{knowls/rcs/rcs.ack.lf}
\input{knowls/rcs/rcs.ack.lfunction}
\input{knowls/rcs/rcs.ack.lfunction.curve}
\input{knowls/rcs/rcs.ack.lfunction.ec}
\input{knowls/rcs/rcs.ack.lfunction.maass}
\input{knowls/rcs/rcs.ack.lfunction.modular}
\input{knowls/rcs/rcs.ack.maass}
\input{knowls/rcs/rcs.ack.maass_rigor}
\input{knowls/rcs/rcs.ack.meetings}
\input{knowls/rcs/rcs.ack.mf.bianchi}
\input{knowls/rcs/rcs.ack.mf.hilbert}
\input{knowls/rcs/rcs.ack.mf.siegel}
\input{knowls/rcs/rcs.ack.modcurve}
\input{knowls/rcs/rcs.ack.nf}
\input{knowls/rcs/rcs.ack.software}
\input{knowls/rcs/rcs.ack.sponsors}
\input{knowls/rcs/rcs.ack.st_group}
\input{knowls/rcs/rcs.cande.artin}
\input{knowls/rcs/rcs.cande.av.fq}
\input{knowls/rcs/rcs.cande.belyi}
\input{knowls/rcs/rcs.cande.character.dirichlet}
\input{knowls/rcs/rcs.cande.cmf}
\input{knowls/rcs/rcs.cande.curve.highergenus.aut}
\input{knowls/rcs/rcs.cande.ec}
\input{knowls/rcs/rcs.cande.ec.q}
\input{knowls/rcs/rcs.cande.g2c}
\input{knowls/rcs/rcs.cande.gg}
\input{knowls/rcs/rcs.cande.groups.abstract}
\input{knowls/rcs/rcs.cande.hgm}
\input{knowls/rcs/rcs.cande.hmsurface}
\input{knowls/rcs/rcs.cande.lattice}
\input{knowls/rcs/rcs.cande.lf}
\input{knowls/rcs/rcs.cande.lfunction}
\input{knowls/rcs/rcs.cande.maass}
\input{knowls/rcs/rcs.cande.maass_rigor}
\input{knowls/rcs/rcs.cande.mf.bianchi}
\input{knowls/rcs/rcs.cande.mf.hilbert}
\input{knowls/rcs/rcs.cande.mf.siegel}
\input{knowls/rcs/rcs.cande.modcurve}
\input{knowls/rcs/rcs.cande.nf}
\input{knowls/rcs/rcs.cande.st_group}
\input{knowls/rcs/rcs.cande.zeros.zeta}
\input{knowls/rcs/rcs.cite.artin}
\input{knowls/rcs/rcs.cite.av.fq}
\input{knowls/rcs/rcs.cite.belyi}
\input{knowls/rcs/rcs.cite.character.dirichlet}
\input{knowls/rcs/rcs.cite.cmf}
\input{knowls/rcs/rcs.cite.curve.highergenus.aut}
\input{knowls/rcs/rcs.cite.ec}
\input{knowls/rcs/rcs.cite.ec.q}
\input{knowls/rcs/rcs.cite.g2c}
\input{knowls/rcs/rcs.cite.gg}
\input{knowls/rcs/rcs.cite.groups.abstract}
\input{knowls/rcs/rcs.cite.hgm}
\input{knowls/rcs/rcs.cite.lf}
\input{knowls/rcs/rcs.cite.lfunction}
\input{knowls/rcs/rcs.cite.maass}
\input{knowls/rcs/rcs.cite.mf.bianchi}
\input{knowls/rcs/rcs.cite.mf.hilbert}
\input{knowls/rcs/rcs.cite.nf}
\input{knowls/rcs/rcs.cite.st_group}
\input{knowls/rcs/rcs.groups.abstract.source}
\input{knowls/rcs/rcs.release.1.3}
\input{knowls/rcs/rcs.rigor.artin}
\input{knowls/rcs/rcs.rigor.av.fq}
\input{knowls/rcs/rcs.rigor.belyi}
\input{knowls/rcs/rcs.rigor.character.dirichlet}
\input{knowls/rcs/rcs.rigor.cmf}
\input{knowls/rcs/rcs.rigor.curve.highergenus.aut}
\input{knowls/rcs/rcs.rigor.ec}
\input{knowls/rcs/rcs.rigor.ec.q}
\input{knowls/rcs/rcs.rigor.g2c}
\input{knowls/rcs/rcs.rigor.gg}
\input{knowls/rcs/rcs.rigor.groups.abstract}
\input{knowls/rcs/rcs.rigor.hgm}
\input{knowls/rcs/rcs.rigor.lattice}
\input{knowls/rcs/rcs.rigor.lf}
\input{knowls/rcs/rcs.rigor.lfunction}
\input{knowls/rcs/rcs.rigor.lfunction.curve}
\input{knowls/rcs/rcs.rigor.lfunction.dirichlet}
\input{knowls/rcs/rcs.rigor.lfunction.ec}
\input{knowls/rcs/rcs.rigor.lfunction.lcalc}
\input{knowls/rcs/rcs.rigor.lfunction.maass}
\input{knowls/rcs/rcs.rigor.lfunction.modular}
\input{knowls/rcs/rcs.rigor.maass}
\input{knowls/rcs/rcs.rigor.maass_rigor}
\input{knowls/rcs/rcs.rigor.mf.bianchi}
\input{knowls/rcs/rcs.rigor.mf.hilbert}
\input{knowls/rcs/rcs.rigor.mf.siegel}
\input{knowls/rcs/rcs.rigor.modcurve}
\input{knowls/rcs/rcs.rigor.modlgal}
\input{knowls/rcs/rcs.rigor.nf}
\input{knowls/rcs/rcs.rigor.st_group}
\input{knowls/rcs/rcs.rigor.zeros.zeta}
\input{knowls/rcs/rcs.source.artin}
\input{knowls/rcs/rcs.source.av.fq}
\input{knowls/rcs/rcs.source.belyi}
\input{knowls/rcs/rcs.source.character.dirichlet}
\input{knowls/rcs/rcs.source.cmf}
\input{knowls/rcs/rcs.source.curve.highergenus.aut}
\input{knowls/rcs/rcs.source.ec}
\input{knowls/rcs/rcs.source.ec.q}
\input{knowls/rcs/rcs.source.g2c}
\input{knowls/rcs/rcs.source.gg}
\input{knowls/rcs/rcs.source.groups.abstract}
\input{knowls/rcs/rcs.source.hgm}
\input{knowls/rcs/rcs.source.hmsurface}
\input{knowls/rcs/rcs.source.lattice}
\input{knowls/rcs/rcs.source.lf}
\input{knowls/rcs/rcs.source.lfunction}
\input{knowls/rcs/rcs.source.lfunction.curve}
\input{knowls/rcs/rcs.source.lfunction.dirichlet}
\input{knowls/rcs/rcs.source.lfunction.ec}
\input{knowls/rcs/rcs.source.lfunction.lcalc}
\input{knowls/rcs/rcs.source.lfunction.maass}
\input{knowls/rcs/rcs.source.lfunction.modular}
\input{knowls/rcs/rcs.source.maass}
\input{knowls/rcs/rcs.source.maass_rigor}
\input{knowls/rcs/rcs.source.mf.bianchi}
\input{knowls/rcs/rcs.source.mf.hilbert}
\input{knowls/rcs/rcs.source.mf.siegel}
\input{knowls/rcs/rcs.source.modcurve}
\input{knowls/rcs/rcs.source.modlgal}
\input{knowls/rcs/rcs.source.nf}
\input{knowls/rcs/rcs.source.st_group}
\input{knowls/rcs/rcs.source.zeros.zeta}
\input{knowls/repn/repn.about}
\input{knowls/repn/repn.highest.weight}
\input{knowls/repn/repn.irrep.gl}
\input{knowls/ring/ring}
\input{knowls/ring/ring.a-field}
\input{knowls/ring/ring.associate}
\input{knowls/ring/ring.characteristic}
\input{knowls/ring/ring.dedekind_domain}
\input{knowls/ring/ring.euclidean_domain}
\input{knowls/ring/ring.field}
\input{knowls/ring/ring.field_of_fractions}
\input{knowls/ring/ring.fractional_ideal}
\input{knowls/ring/ring.ideal}
\input{knowls/ring/ring.integral}
\input{knowls/ring/ring.integral_domain}
\input{knowls/ring/ring.integrally_closed}
\input{knowls/ring/ring.irreducible}
\input{knowls/ring/ring.maximal_ideal}
\input{knowls/ring/ring.noetherian}
\input{knowls/ring/ring.prime_ideal}
\input{knowls/ring/ring.principal_fractional_ideal}
\input{knowls/ring/ring.principal_ideal_domain}
\input{knowls/ring/ring.unique_factorization_domain}
\input{knowls/ring/ring.unit}
\input{knowls/ring/ring.zero_divisor}
\input{knowls/sage/sage}
\input{knowls/sage/sage.cell}
\input{knowls/sage/sage.test}
\input{knowls/sh/sh.disc}
\input{knowls/sh/sh.level}
\input{knowls/shimcurve/shimcurve}
\input{knowls/shimcurve/shimcurve.b}
\input{knowls/shimcurve/shimcurve.discb}
\input{knowls/shimcurve/shimcurve.disco}
\input{knowls/shimcurve/shimcurve.enhanced_group}
\input{knowls/shimcurve/shimcurve.genus}
\input{knowls/shimcurve/shimcurve.gonality}
\input{knowls/shimcurve/shimcurve.index}
\input{knowls/shimcurve/shimcurve.invariants}
\input{knowls/shimcurve/shimcurve.level}
\input{knowls/shimcurve/shimcurve.nrdmu}
\input{knowls/shimcurve/shimcurve.order}
\input{knowls/shimcurve/shimcurve.polarized_order}
\input{knowls/shimcurve/shimcurve.quaternion_algebra}
\input{knowls/specialfunction/specialfunction.gamma}
\input{knowls/specialfunction/specialfunction.k_bessel}
\input{knowls/specialfunction/specialfunction.poincare}
\input{knowls/st_group/st_group.ambient}
\input{knowls/st_group/st_group.component_group}
\input{knowls/st_group/st_group.components}
\input{knowls/st_group/st_group.connected}
\input{knowls/st_group/st_group.data}
\input{knowls/st_group/st_group.definition}
\input{knowls/st_group/st_group.degree}
\input{knowls/st_group/st_group.embedding}
\input{knowls/st_group/st_group.first_a2_moment}
\input{knowls/st_group/st_group.fourth_trace_moment}
\input{knowls/st_group/st_group.generators}
\input{knowls/st_group/st_group.hodge_circle}
\input{knowls/st_group/st_group.identity_component}
\input{knowls/st_group/st_group.index}
\input{knowls/st_group/st_group.invariants}
\input{knowls/st_group/st_group.label}
\input{knowls/st_group/st_group.moment_matrix}
\input{knowls/st_group/st_group.moment_simplex}
\input{knowls/st_group/st_group.moments}
\input{knowls/st_group/st_group.name}
\input{knowls/st_group/st_group.probabilities}
\input{knowls/st_group/st_group.rational}
\input{knowls/st_group/st_group.real_dimension}
\input{knowls/st_group/st_group.search_input}
\input{knowls/st_group/st_group.second_trace_moment}
\input{knowls/st_group/st_group.subgroups}
\input{knowls/st_group/st_group.subsupgroups}
\input{knowls/st_group/st_group.summary}
\input{knowls/st_group/st_group.supgroups}
\input{knowls/st_group/st_group.symplectic_form}
\input{knowls/st_group/st_group.trace_moments}
\input{knowls/st_group/st_group.trace_zero_density}
\input{knowls/st_group/st_group.usp}
\input{knowls/st_group/st_group.weight}
\input{knowls/stats/stats.buckets}
\input{knowls/stats/stats.proportions}
\input{knowls/stats/stats.totals}
\input{knowls/test/test.knowlparam}
\input{knowls/test/test.suggestions}
\input{knowls/test/test.text}
\input{knowls/upload/upload.modcurve.name_or_label}
\input{knowls/upload/upload.multi_knowl}
\input{knowls/users/users.edgarcosta.note}
\input{knowls/users/users.jhs.note}
\input{knowls/varieties/varieties.about}
\input{knowls/zeros/zeros.first.contents}


\section{Number fields}
\input{knowls/nf/nf}
\input{knowls/nf/nf.abelian}
\input{knowls/nf/nf.abs_discriminant}
\input{knowls/nf/nf.absolute_value}
\input{knowls/nf/nf.algebraic_closure}
\input{knowls/nf/nf.arithmetically_equivalent}
\input{knowls/nf/nf.assuming_grh}
\input{knowls/nf/nf.class_number}
\input{knowls/nf/nf.class_number_formula}
\input{knowls/nf/nf.cm_field}
\input{knowls/nf/nf.complex_embedding}
\input{knowls/nf/nf.conductor}
\input{knowls/nf/nf.defining_polynomial}
\input{knowls/nf/nf.defining_polynomial.normalization}
\input{knowls/nf/nf.degree}
\input{knowls/nf/nf.dirichlet_group}
\input{knowls/nf/nf.discriminant}
\input{knowls/nf/nf.discriminant_root_field}
\input{knowls/nf/nf.elkies}
\input{knowls/nf/nf.embedding}
\input{knowls/nf/nf.field.data}
\input{knowls/nf/nf.field.link}
\input{knowls/nf/nf.field.missing}
\input{knowls/nf/nf.field_is}
\input{knowls/nf/nf.frobenius_cycle_types}
\input{knowls/nf/nf.fundamental_units}
\input{knowls/nf/nf.galois_closure}
\input{knowls/nf/nf.galois_group}
\input{knowls/nf/nf.galois_group.data}
\input{knowls/nf/nf.galois_group.gmodule}
\input{knowls/nf/nf.galois_group.gmodule_v4_type_i_i}
\input{knowls/nf/nf.galois_group.gmodule_v4_type_i_ii}
\input{knowls/nf/nf.galois_group.gmodule_v4_type_i_iii}
\input{knowls/nf/nf.galois_group.gmodule_v4_type_ii_i}
\input{knowls/nf/nf.galois_group.gmodule_v4_type_ii_ii}
\input{knowls/nf/nf.galois_group.gmodule_v4_type_iii}
\input{knowls/nf/nf.galois_group.gmodule_v4_type_iv}
\input{knowls/nf/nf.galois_group.name}
\input{knowls/nf/nf.galois_root_discriminant}
\input{knowls/nf/nf.galois_search}
\input{knowls/nf/nf.generator}
\input{knowls/nf/nf.ideal.label.hmf}
\input{knowls/nf/nf.ideal_class_group}
\input{knowls/nf/nf.ideal_labels}
\input{knowls/nf/nf.index}
\input{knowls/nf/nf.inessential_prime}
\input{knowls/nf/nf.integral}
\input{knowls/nf/nf.integral_basis}
\input{knowls/nf/nf.intermediate_fields}
\input{knowls/nf/nf.invariants}
\input{knowls/nf/nf.is_galois}
\input{knowls/nf/nf.isogeny_primes}
\input{knowls/nf/nf.label}
\input{knowls/nf/nf.local_algebra}
\input{knowls/nf/nf.maximal_cm_subfield}
\input{knowls/nf/nf.minimal_polynomial}
\input{knowls/nf/nf.minimal_sibling}
\input{knowls/nf/nf.monogenic}
\input{knowls/nf/nf.monomial_order}
\input{knowls/nf/nf.multiplicative_gal_module}
\input{knowls/nf/nf.narrow_class_group}
\input{knowls/nf/nf.narrow_class_number}
\input{knowls/nf/nf.nickname}
\input{knowls/nf/nf.order}
\input{knowls/nf/nf.padic_completion}
\input{knowls/nf/nf.padic_completion.search}
\input{knowls/nf/nf.place}
\input{knowls/nf/nf.polredabs}
\input{knowls/nf/nf.poly_discriminant}
\input{knowls/nf/nf.prime}
\input{knowls/nf/nf.primitive_element}
\input{knowls/nf/nf.ramified_primes}
\input{knowls/nf/nf.rank}
\input{knowls/nf/nf.real_embedding}
\input{knowls/nf/nf.reflex_field}
\input{knowls/nf/nf.reflex_reflex_field}
\input{knowls/nf/nf.regulator}
\input{knowls/nf/nf.relative_class_number}
\input{knowls/nf/nf.ring_of_integers}
\input{knowls/nf/nf.root_discriminant}
\input{knowls/nf/nf.search_input}
\input{knowls/nf/nf.separable}
\input{knowls/nf/nf.separable_algebra}
\input{knowls/nf/nf.serre_odlyzko_bound}
\input{knowls/nf/nf.sextic_twin}
\input{knowls/nf/nf.sibling}
\input{knowls/nf/nf.signature}
\input{knowls/nf/nf.spectrum_ring_of_integers}
\input{knowls/nf/nf.stem_field}
\input{knowls/nf/nf.torsion}
\input{knowls/nf/nf.totally_imaginary}
\input{knowls/nf/nf.totally_positive}
\input{knowls/nf/nf.totally_real}
\input{knowls/nf/nf.unit_group}
\input{knowls/nf/nf.unramified_prime}
\input{knowls/nf/nf.weil_height}
\input{knowls/nf/nf.weil_polynomial}
\input{knowls/nf/nf.zk_index}


\section{Elliptic curves}
\input{knowls/ec/ec}
\input{knowls/ec/ec.additive_reduction}
\input{knowls/ec/ec.analytic_sha_order}
\input{knowls/ec/ec.bad_reduction}
\input{knowls/ec/ec.base_change}
\input{knowls/ec/ec.bsdconjecture}
\input{knowls/ec/ec.canonical_height}
\input{knowls/ec/ec.complex_multiplication}
\input{knowls/ec/ec.conductor}
\input{knowls/ec/ec.conductor_label}
\input{knowls/ec/ec.conductor_valuation}
\input{knowls/ec/ec.congruent_number}
\input{knowls/ec/ec.congruent_number_curve}
\input{knowls/ec/ec.congruent_number_problem}
\input{knowls/ec/ec.curve_label}
\input{knowls/ec/ec.discriminant}
\input{knowls/ec/ec.discriminant_norm}
\input{knowls/ec/ec.discriminant_valuation}
\input{knowls/ec/ec.endomorphism}
\input{knowls/ec/ec.endomorphism_ring}
\input{knowls/ec/ec.galois_image_search}
\input{knowls/ec/ec.galois_rep}
\input{knowls/ec/ec.galois_rep_adelic_image}
\input{knowls/ec/ec.galois_rep_elladic_image}
\input{knowls/ec/ec.galois_rep_modell_image}
\input{knowls/ec/ec.geom_endomorphism_ring}
\input{knowls/ec/ec.global_minimal_model}
\input{knowls/ec/ec.good_ordinary_reduction}
\input{knowls/ec/ec.good_reduction}
\input{knowls/ec/ec.good_supersingular_reduction}
\input{knowls/ec/ec.integral_model}
\input{knowls/ec/ec.invariants}
\input{knowls/ec/ec.isogeny}
\input{knowls/ec/ec.isogeny_class}
\input{knowls/ec/ec.isogeny_class_degree}
\input{knowls/ec/ec.isogeny_graph}
\input{knowls/ec/ec.isogeny_matrix}
\input{knowls/ec/ec.isogenygraph.legend}
\input{knowls/ec/ec.isomorphism}
\input{knowls/ec/ec.iwasawa_invariants}
\input{knowls/ec/ec.j_invariant}
\input{knowls/ec/ec.j_invariant_denominator_valuation}
\input{knowls/ec/ec.kodaira_symbol}
\input{knowls/ec/ec.lambda_invariant}
\input{knowls/ec/ec.local_data}
\input{knowls/ec/ec.local_minimal_discriminant}
\input{knowls/ec/ec.local_minimal_model}
\input{knowls/ec/ec.local_root_number}
\input{knowls/ec/ec.maximal_elladic_galois_rep}
\input{knowls/ec/ec.maximal_galois_rep}
\input{knowls/ec/ec.minimal_discriminant}
\input{knowls/ec/ec.mordell_weil_group}
\input{knowls/ec/ec.mordell_weil_theorem}
\input{knowls/ec/ec.mu_invariant}
\input{knowls/ec/ec.multiplicative_reduction}
\input{knowls/ec/ec.mw_generators}
\input{knowls/ec/ec.nonsplit_multiplicative_reduction}
\input{knowls/ec/ec.obstruction_class}
\input{knowls/ec/ec.padic_tate_module}
\input{knowls/ec/ec.period}
\input{knowls/ec/ec.potential_good_reduction}
\input{knowls/ec/ec.q}
\input{knowls/ec/ec.q.abc_quality}
\input{knowls/ec/ec.q.analytic_rank}
\input{knowls/ec/ec.q.analytic_sha_order}
\input{knowls/ec/ec.q.analytic_sha_value}
\input{knowls/ec/ec.q.bsd_invariants}
\input{knowls/ec/ec.q.bsdconjecture}
\input{knowls/ec/ec.q.canonical_height}
\input{knowls/ec/ec.q.conductor}
\input{knowls/ec/ec.q.cremona_label}
\input{knowls/ec/ec.q.discriminant}
\input{knowls/ec/ec.q.endomorphism_ring}
\input{knowls/ec/ec.q.faltings_height}
\input{knowls/ec/ec.q.faltings_ratio}
\input{knowls/ec/ec.q.frey}
\input{knowls/ec/ec.q.integral_points}
\input{knowls/ec/ec.q.invariants}
\input{knowls/ec/ec.q.j_invariant}
\input{knowls/ec/ec.q.kodaira_symbol}
\input{knowls/ec/ec.q.lmfdb_label}
\input{knowls/ec/ec.q.manin_constant}
\input{knowls/ec/ec.q.minimal_twist}
\input{knowls/ec/ec.q.minimal_weierstrass_equation}
\input{knowls/ec/ec.q.modular_degree}
\input{knowls/ec/ec.q.modular_form}
\input{knowls/ec/ec.q.modular_parametrization}
\input{knowls/ec/ec.q.naive_height}
\input{knowls/ec/ec.q.optimal}
\input{knowls/ec/ec.q.period_lattice}
\input{knowls/ec/ec.q.real_period}
\input{knowls/ec/ec.q.reduction_type}
\input{knowls/ec/ec.q.regulator}
\input{knowls/ec/ec.q.search_input}
\input{knowls/ec/ec.q.semistable}
\input{knowls/ec/ec.q.serre_invariants}
\input{knowls/ec/ec.q.special_value}
\input{knowls/ec/ec.q.szpiro_ratio}
\input{knowls/ec/ec.q.tamagawa_numbers}
\input{knowls/ec/ec.q.torsion_growth}
\input{knowls/ec/ec.q.torsion_subgroup}
\input{knowls/ec/ec.q.weil_height}
\input{knowls/ec/ec.q_curve}
\input{knowls/ec/ec.rank}
\input{knowls/ec/ec.reduction}
\input{knowls/ec/ec.reduction_type}
\input{knowls/ec/ec.regulator}
\input{knowls/ec/ec.ring}
\input{knowls/ec/ec.rouse_label}
\input{knowls/ec/ec.scheme}
\input{knowls/ec/ec.search_input}
\input{knowls/ec/ec.semi_global_minimal_model}
\input{knowls/ec/ec.semistable}
\input{knowls/ec/ec.simple_equation}
\input{knowls/ec/ec.special_value}
\input{knowls/ec/ec.split_multiplicative_reduction}
\input{knowls/ec/ec.tamagawa_number}
\input{knowls/ec/ec.torsion_order}
\input{knowls/ec/ec.torsion_subgroup}
\input{knowls/ec/ec.twists}
\input{knowls/ec/ec.weierstrass_coeffs}
\input{knowls/ec/ec.weierstrass_isomorphism}


\section{Modular forms}
\input{knowls/cmf/cmf}
\input{knowls/cmf/cmf.analytic_conductor}
\input{knowls/cmf/cmf.analytic_rank}
\input{knowls/cmf/cmf.artin_field}
\input{knowls/cmf/cmf.artin_image}
\input{knowls/cmf/cmf.atkin-lehner}
\input{knowls/cmf/cmf.atkin_lehner_dims}
\input{knowls/cmf/cmf.bad_prime}
\input{knowls/cmf/cmf.character}
\input{knowls/cmf/cmf.cm_form}
\input{knowls/cmf/cmf.coefficient_field}
\input{knowls/cmf/cmf.coefficient_ring}
\input{knowls/cmf/cmf.congruence_subgroup}
\input{knowls/cmf/cmf.cusp_form}
\input{knowls/cmf/cmf.decomposition.new.gamma0chi}
\input{knowls/cmf/cmf.decomposition.new.gamma1}
\input{knowls/cmf/cmf.decomposition.old.gamma0chi}
\input{knowls/cmf/cmf.decomposition.old.gamma1}
\input{knowls/cmf/cmf.defining_polynomial}
\input{knowls/cmf/cmf.dim_decomposition}
\input{knowls/cmf/cmf.dimension}
\input{knowls/cmf/cmf.dimension_galois_orbit}
\input{knowls/cmf/cmf.display_dim}
\input{knowls/cmf/cmf.distinguishing_primes}
\input{knowls/cmf/cmf.dualform}
\input{knowls/cmf/cmf.eisenstein}
\input{knowls/cmf/cmf.embedding}
\input{knowls/cmf/cmf.embedding_format}
\input{knowls/cmf/cmf.embedding_label}
\input{knowls/cmf/cmf.eta_quotient}
\input{knowls/cmf/cmf.fouriercoefficients}
\input{knowls/cmf/cmf.fricke}
\input{knowls/cmf/cmf.galois_conjugate}
\input{knowls/cmf/cmf.galois_orbit}
\input{knowls/cmf/cmf.galois_representation}
\input{knowls/cmf/cmf.hecke_galois_orbit}
\input{knowls/cmf/cmf.hecke_kernels}
\input{knowls/cmf/cmf.hecke_operator}
\input{knowls/cmf/cmf.hecke_orbit}
\input{knowls/cmf/cmf.hecke_ring_generators}
\input{knowls/cmf/cmf.heckecharpolys}
\input{knowls/cmf/cmf.include_all_spaces}
\input{knowls/cmf/cmf.inner_twist}
\input{knowls/cmf/cmf.inner_twist_count}
\input{knowls/cmf/cmf.inner_twist_group}
\input{knowls/cmf/cmf.inner_twist_multiplicity}
\input{knowls/cmf/cmf.inner_twist_proved}
\input{knowls/cmf/cmf.label}
\input{knowls/cmf/cmf.level}
\input{knowls/cmf/cmf.lfunction}
\input{knowls/cmf/cmf.maximal}
\input{knowls/cmf/cmf.minimal}
\input{knowls/cmf/cmf.minimal_twist}
\input{knowls/cmf/cmf.minus_space}
\input{knowls/cmf/cmf.newform}
\input{knowls/cmf/cmf.newform_subspace}
\input{knowls/cmf/cmf.newspace}
\input{knowls/cmf/cmf.newspaces}
\input{knowls/cmf/cmf.nk2}
\input{knowls/cmf/cmf.nontrivial_twist}
\input{knowls/cmf/cmf.oldspace}
\input{knowls/cmf/cmf.petersson_scalar_product}
\input{knowls/cmf/cmf.picture_description}
\input{knowls/cmf/cmf.plus_space}
\input{knowls/cmf/cmf.projective_field}
\input{knowls/cmf/cmf.projective_image}
\input{knowls/cmf/cmf.q-expansion}
\input{knowls/cmf/cmf.relative_dimension}
\input{knowls/cmf/cmf.rm_form}
\input{knowls/cmf/cmf.root}
\input{knowls/cmf/cmf.satake_angles}
\input{knowls/cmf/cmf.satake_parameters}
\input{knowls/cmf/cmf.sato_tate}
\input{knowls/cmf/cmf.search_input}
\input{knowls/cmf/cmf.self_twist}
\input{knowls/cmf/cmf.self_twist_field}
\input{knowls/cmf/cmf.selfdual}
\input{knowls/cmf/cmf.shimura_correspondence}
\input{knowls/cmf/cmf.sort_order}
\input{knowls/cmf/cmf.space}
\input{knowls/cmf/cmf.space_trace_form}
\input{knowls/cmf/cmf.stark_unit}
\input{knowls/cmf/cmf.statistics_extent}
\input{knowls/cmf/cmf.sturm_bound}
\input{knowls/cmf/cmf.sturm_bound_gamma1}
\input{knowls/cmf/cmf.subspaces}
\input{knowls/cmf/cmf.trace_bound}
\input{knowls/cmf/cmf.trace_form}
\input{knowls/cmf/cmf.twist}
\input{knowls/cmf/cmf.twist_minimal}
\input{knowls/cmf/cmf.twist_multiplicity}
\input{knowls/cmf/cmf.weight}

