% In this file you should put the actual content of the blueprint.
% It will be used both by the web and the print version.
% It should *not* include the \begin{document}
%
% If you want to split the blueprint content into several files then
% the current file can be a simple sequence of \input. Otherwise It
% can start with a \section or \chapter for instance.



\section{Number Fields}

\begin{itemize}
    \item label (degree, (real) signature, abs value of discriminant, index?)
    \item root discriminant
    \item Galois root discriminant
    \item ramified primes
    \item discriminant root field
    \item automorphism group
    \item monogeneric
    \item index
    \item inessential primes
    \item torsion generator
    \item fundamental units
    \item regulator
    \item itermediate fields
    \item sibling fields
    \item frobenius cycle type
    \item local algebra for ramified primes (maybe, depends on local fields stuff).
    \end{itemize}


\section{Modular forms}

\begin{itemize}
    \item label (level, weight, galois orbit of dirichlet character, label of galois orbit of newform, corey label, relative dimension)
    \item coefficient field
    \item Character
    \item Hecke operators
    \item newform/old forms
    \item Petersson inner product
    \item L-function self dual
    \item analytic conductor
    \item dimension
    \item Fricke sign/ Atkin-Lehner signs
    \item inner twists
\end{itemize}

\section{Elliptic curves}

\begin{itemize}
    \item LMFDB label (and maybe also Cremona label): Conductor, isogeny class label and isomorphism class index
    \item abc quality
    \item rank
    \item torsion order



\end{itemize}
